\documentclass[a4paper, 14pt]{extreport}

\usepackage{extsizes}
\usepackage[left=3cm,right=1.5cm,top=1.25cm,bottom=2cm]{geometry}

\usepackage[english,russian]{babel}
\usepackage[T1]{fontenc}
\usepackage[utf8]{inputenc}
%\usepackage{fontspec}

\usepackage{ragged2e}
\usepackage{microtype}

\justifying
\sloppy
\tolerance=1000
\hyphenpenalty=10000
\emergencystretch=3em

\usepackage{calc}

%\setmainfont{Helvetica Neue}
%\setmonofont{Fira Code}

\usepackage{graphicx}

\usepackage[tableposition=top]{caption}
\usepackage{subcaption}
\DeclareCaptionLabelFormat{gostfigure}{Рисунок #2}
\DeclareCaptionLabelFormat{gosttable}{Таблица #2}
\DeclareCaptionLabelSeparator{gost}{~---~}
\captionsetup{labelsep=gost}
\captionsetup[figure]{labelformat=gostfigure}
\captionsetup[table]{labelformat=gosttable}
\renewcommand{\thesubfigure}{\asbuk{subfigure}}

\usepackage{amsmath}
\usepackage{tabu}
\usepackage{longtable}
\pagestyle{empty}
% Document
\begin{document}

    \section{Введение}

    В рамках данной работы был построен и оценен метод для прогнозирования поведения пользователей, рассмотрена
    структура решения, оценка качества прогноза.

    Предложенный метод может быть использован при ограничении вычислительных возможностей рабочей станции, а также
    для применения на малых выборках.
    Предложенный алгоритм представляет собой комбинацию известных математических
    техник для создания выборки, прогнозирования одномерных временных рядов и проведения калиборвки.

    В качестве исходных данных был использован набор данных, представляющий собой анонимизированную статистику
    взаимодействия пользователей.

    \subsection*{Описание задачи}

    Целью данного исследования является построения алгоритма, позволявщего с достаточной точностью качественно
    предсказывать взаимодействие пользователя с рекламными элементами интернет страниц.
    <Следом>> данных взаимодействий является собраная рекламным сервером информация, представляющая собой набор
    технических метаданных, куки данные пользователя и данные, собранные при помощи сторонних интеграций.

    Каждое взаимодействие пользователя $U_i$ с рекламным элементом $A_j$ в момент времени $\tau_k$ можно представить
    в виде вектора $\mathbf{U_i}(\tau_t)$, элементами которого являются признаки пользователя, значения которых выбраны
    из числа допустимых значений.
    \begin{equation}
        \mathbf{U_i}\left(\tau_k\right) = \left( u_1, u_2, \dots, u_n \right), n = \overline{1, N}, m = \overline{1, M}
    \end{equation} где $N$ -- число признаков, $M$ -- число уникальных взаимодействующих пользователей.

    Задача данного исследования, на основе известных взаимодействий пользователей
    \begin{equation}
        \left\{ \mathbf{U_i}\left(\tau_k\right)\right\},
        \forall k \in \left(0, T\right),
        \forall i \in \left( 0, M \right)
    \end{equation}
    предсказать поведение пользователей $\mathbf{\widehat{U}_i}\left(\tau_k\right)$ на горизонте времени
    $\widehat{T} > T$, иными словами, найти функцию $F$ такую, что
    \begin{equation}
        F\left( \mathbf{U_i}, \tau \right)_{
        \forall i \in (0, M),
        \forall k \in \left(0, T\right),
        \forall \tau \in \left( T, \widehat{T} \right)}
        = \mathbf{\widehat{U}_i}\left(\tau_k\right).
    \end{equation}

    \subsection*{Описание алгоритма}

    Особенностью взаимодействия пользователей в сети интернет является их возможное условное деление пользователей
    на 2 группы: первая из них является <<постоянным>> пользователем интернет ресурса (т.е. пользователь
    взаимодействовал с элементами рекламы несколько раз за выбранных период), другая же группа являются <<уходящими>>
    пользователям, т.е. взаимодействующими единожды.

    Данная особенность позволяет полагать, что можно использовать статистику взаимодействий уникальных пользователей и
    отобразить ее напрямую при помощи некоторого численного значения (калибровать, подобрать вес, который бы отражал
    влияния пользователя их прошлого во взаимодействие пользователей из будущего).

    Исходя из области использования данных, интересными для калибровки представляются две метрики: метрика количества
    взаимодействий пользователей с элементами интернет рекламы за период и метрика количества уникальных пользователей,
    взаимодействующих с выбранным рекламным элементом за период.

    \paragraph{Количество взаимодействий пользователей с элементами интернет рекламы за период.} Пусть дана выборка
    $\left\{ \mathbf{U_i} \left( \tau_k \right) \right\}$ за период $T$, и пусть $u_j$ -- признак пользователя, который
    показывает на какой интернет странице прошло взаимодействие, тогда $\text{Imps}\left( U_i \right)$ -- взаимодействие
    пользователя $U_i$ c рекламным элементом $A_j$.

    Функция взаимодействий представима в следующем виде:

    \begin{equation}
        \text{Imps}\left( U_i, A_j, \tau_k \right) =
        \begin{cases}
            1, u_i = A_j \\
            0, u_i \neq A_j
        \end{cases}
    \end{equation}

    Тогда количество взаимодействия выбранного пользователя $U_i$ с выбранным рекламным элементом $A_j$ за период
    времени $T$ определяется следующим образом:

    \begin{equation}
        \text{Imps}_{\left.U_i\right\|_{u_i = A_j}} = \sum_{k = 0}^T \text{Imps}\left( U_i, A_j, \tau_k \right)
    \end{equation}

    А метрика количества взаимодействий пользователей с элементами интернет рекламы за период,
    получается следущим образом:

    \begin{equation}
        \text{Imps}_{u_i = A_j} = \sum_{i=0}^M \sum_{k = 0}^T \text{Imps}\left( U_i, A_j, \tau_k \right)
    \end{equation}

\end{document}