\documentclass[a4paper, 14pt]{extreport}

\usepackage{extsizes}
\usepackage[left=3cm,right=1.5cm,top=1.25cm,bottom=2cm]{geometry}

\usepackage[english,russian]{babel}
\usepackage[T1]{fontenc}
\usepackage[utf8]{inputenc}
%\usepackage{fontspec}

\usepackage{ragged2e}
\usepackage{microtype}

\justifying
\sloppy
\tolerance=1000
\hyphenpenalty=10000
\emergencystretch=3em

\usepackage{calc}

%\setmainfont{Helvetica Neue}
%\setmonofont{Fira Code}

\usepackage{graphicx}

\usepackage[tableposition=top]{caption}
\usepackage{subcaption}
\DeclareCaptionLabelFormat{gostfigure}{Рисунок #2}
\DeclareCaptionLabelFormat{gosttable}{Таблица #2}
\DeclareCaptionLabelSeparator{gost}{~---~}
\captionsetup{labelsep=gost}
\captionsetup[figure]{labelformat=gostfigure}
\captionsetup[table]{labelformat=gosttable}
\renewcommand{\thesubfigure}{\asbuk{subfigure}}

\usepackage{amsmath}
\usepackage{tabu}
\usepackage{longtable}
\pagestyle{empty}
% Document
\begin{document}

    \section{Введение}

    В рамках данной работы был построен и оценен метод для прогнозирования поведения пользователей, рассмотрена
    структура решения, оценка качества прогноза.

    Предложенный метод может быть использован при ограничении вычислительных возможностей рабочей станции, а также
    для применения на малых выборках.
    Предложенный алгоритм представляет собой комбинацию известных математических
    техник для создания выборки, прогнозирования одномерных временных рядов и проведения калиборвки.

    В качестве исходных данных был использован набор данных, представляющий собой анонимизированную статистику
    взаимодействия пользователей.

    \subsection*{Описание задачи}

    Целью данного исследования является построения алгоритма, позволявщего с достаточной точностью качественно
    предсказывать взаимодействие пользователя с рекламными элементами интернет страниц.
    <Следом>> данных взаимодействий является собраная рекламным сервером информация, представляющая собой набор
    технических метаданных, куки данные пользователя и данные, собранные при помощи сторонних интеграций.

    Каждое взаимодействие пользователя в момент времени $t_k$ можно представить в виде вектора $\mathbf{U_i}(\tau_k)$,
    элементами которого являются признаки пользователя (техническая информация, собранная рекламным сервером).
    \begin{equation}
        \mathbf{U_i}\left(t_k\right) =
        \left( u_1, u_2, \dots, u_n, t_k \right),
        i = \overline{1, M}, n = \overline{1, N}, k = \overline{1, K},
    \end{equation} где $M$ -- общее число взаимодействий, $N$ -- число признаков, $K$ -- число временных отсчетов.

    Задача данного исследования, на основе известных взаимодействий пользователей
    (множество наблюдений $\left\{ U_i \left(t_k\right) \right\}$)
    \begin{equation}
        \left\{ \mathbf{U_i}\left(t_k\right)\right\},
        \forall k \in \left(0, T\right),
        \forall i \in \left( 0, M \right)
    \end{equation}
    предсказать взаимодействие пользователей $\mathbf{\widehat{U}_i}\left(t_k\right)$ на горизонте времени
    $\widehat{T} > t_K$, иными словами, найти функцию $F$ такую, что
    \begin{equation}
        F\left( \mathbf{U_i}\left(t_k \right), T \right)_{
        \forall i \in (0, M),
        \forall k \in \left(0, K\right)}
        = \mathbf{\widehat{U}_i}\left(T\right).
    \end{equation}

    \subsection*{Описание алгоритма}

    Основная идея авторов работы заключается в комбинации существующего метода прогнозирования Naive Approach (наивное
    прогнозирование)
    \begin{equation}
        F\left( \mathbf{U_i}\left(t_k + h \right) \right)
        = \mathbf{\widehat{U}_i}\left(t_k\right)
    \end{equation}
    и метода калибровки выборки
    \begin{equation}
        F\left( \mathbf{U_i}\left(t_k + h \right) \right)
        = w_{ik} \cdot \mathbf{{U}_i}\left(t_k\right)
    \end{equation}
    \begin{equation}
        w_{ik} = G\left( \mathbf{U_i}\left(t_k + h \right) \right)
    \end{equation}.

    Для определения способа калибровки наивного прогноза авторами рассмотрено поведение пользователей интернет сети.
    Особенностью этого поведения является возможное условное деление на 2 группы: первая из них является <<постоянным>>
    пользователем интернет ресурса (т.е. пользователь взаимодействовал с элементами рекламы несколько раз за выбранный
    период), другая же группа являются <<уходящими>> пользователям, т.е. взаимодействующими единожды.

    \paragraph{Постоянные и уходящие пользователи.} Предположим, мы рассматриваем взаимодействия пользователей с
    рекламным элементом $A$ на отрезке времени $\left[ T_0, T_1 \right]$ и $u_i$ -- признак взаимодействия,
    который показывает какой пользователь его совершил, $u_j$ -- признак взаимодействия, который показывает на какой
    интернет странице оно произошло, тогда если мы введем функцию $\text{Imps}\left(U\left(t_k\right)\right)$, которая
    будет возвращать количество взаимодействий для выбранного пользователя $U$, то постоянными пользователями будут
    называться такие пользователи, для которых
    \begin{equation}
        \left.\text{Imps}\left(U\right)\right\|_{u_j = U} > 1, t_k \in \left[ T_0, T_1 \right],
    \end{equation}
    а уходящими такие, что
    \begin{equation}
        \left.\text{Imps}\left(U\right)\right\|_{u_j = U} \leq 1, t_k \in \left[ T_0, T_1 \right]
    \end{equation}

    Данное поведение, как уже описано выше, возможно описать двумя метриками: метрикой количества взаимодействий и
    метрикой количества уникальных взаимодействий.

    \paragraph{Метрика количества взаимодействий.} Пусть дана выборка взаимодействий пользователей
    $\left\{\mathbf{U_i}\left(t_k\right)\right\}$ за период времени $[T_0, T_1]$, $u_i$ -- признак взаимодействия,
    который показывает какой пользователь его совершил, $u_j$ -- признак взаимодействия, который показывает на какой
    интернет странице оно произошло, тогда $\left. \text{Imps} \left( U_i (t_k) \right) \right\|_{u_i, u_j}$ --
    взаимодействие пользователя $u_i$ c рекламным элементом $u_j$ в момент времени $t_k$.

    Также данная функция представима в следующем виде:

    \begin{equation}
        \left. \text{Imps} \left( U_i (t_k) \right) \right\|_{u_i = U, u_j = A} =
        \begin{cases}
            1, u_i = U \wedge u_j = A  \\
            0, u_i \neq U \vee u_j \neq A
        \end{cases}, \forall i \in (0, M)
    \end{equation}

    Метрика количества взаимодействий определяется на временном интервале $[T_0, T_1]$
    \begin{equation}
        \text{Imps} \left( U, A, \left[ T_0, T_1 \right] \right) =
        \sum \limits_{t=T_0}^{T_1} \left. \text{Imps} \left( U_i (t) \right) \right\|_{u_i = U, u_j = A}
    \end{equation}

    Визуализация

    \paragraph{Метрика количества взаимодействий уникальных пользователей.} Пусть дана выборка взаимодействий
    пользователей $\left\{\mathbf{U_i}\left(t_k\right)\right\}$ за период времени $[T_0, T_1]$,  $u_j$ -- признак
    взаимодействия, который показывает на какой интернет странице оно произошло, тогда
    $\left.\text{Distinct}\left( U_i \right)\right\|_{u_i, u_j}$ -- количество взаимодействий уникальных пользователей
    с рекламным элементом $u_j$.

    Тогда количество взаимодействия выбранного пользователя $U_i$ с выбранным рекламным элементом $A_j$ за период
    времени $T$ определяется следующим образом:

    \begin{equation}
        \text{Imps}_{\left.U_i\right\|_{u_i = A_j}} = \sum_{k = 0}^T \text{Imps}\left( U_i, A_j, \tau_k \right)
    \end{equation}

    А метрика количества взаимодействий пользователей с элементами интернет рекламы за период,
    получается следущим образом:

    \begin{equation}
        \text{Imps}_{u_i = A_j} = \sum_{i=0}^M \sum_{k = 0}^T \text{Imps}\left( U_i, A_j, \tau_k \right)
    \end{equation}

\end{document}