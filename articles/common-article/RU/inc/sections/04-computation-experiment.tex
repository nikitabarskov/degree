\section{Результаты численного эксперимента}
\subsection{Описание исходного набора данных}
В качестве исходных данных для численного эксперимента был использован
набор данных, представляющий собой анонимизированный трафик пользователей
крупной сети сайтов. В представлена сводная характеристика используемых
данных.
\begin{longtabu} to \textwidth {|X|X|}
    \caption{Описание исходных данных}
    \label{tab:dataset-description}
    \endhead
    \rowfont[c]{\bfseries}
    \hline
    Параметр набора данных & Значение \\
    \hline
    Временной охват & 1 марта 2018 года --- 1 марта 2019 года \\
    \hline
    Число уникальных рекламных элементов & \\
    \hline
    Число уникальных пользователей & \\
    \hline
    Общее число взаимодействий & \\
    \hline
\end{longtabu}
\subsection{Численный эксперимент}
При оценке данного метода были проведены эксперименты по оценке качества прогноза 
на различные временные промежутки.
\subsubsection{Прогнозирование на 7 дней}
\begin{figure}[ht]
    \centering
    \begin{minipage}{0.45\textwidth}    
        \centering
        \begin{tikzpicture}[font=\scriptsize]
            \begin{axis}[
                unit vector ratio*=1 1 1,
                width=\linewidth,
                xlabel=Предсказанное значение,
                ylabel=Наблюдаемое значение,
                legend style={
                    at={(0.5,1.2)},
                    anchor=south,
                    nodes={right},
                    font=\scriptsize\mystrut},
                legend columns=2,
                xmax=1e7,
                ymin=-0.01,
                xmin=-0.01,
                ymax=1e7]
            \addplot+[
                only marks,
                mark=*,
                opacity=0.3]
            table {\algoforecastoneweek};
            \addplot+[
                only marks,
                mark=*,
                opacity=0.3]
            table {\naiveforecastoneweek};
            \addplot[
                transparent,
                no markers,
                name path=A,
                domain=0:1e8
            ]{1.1*x};
            \addplot[
                transparent,
                no markers,
                name path=B,
                domain=0:1e8
            ]{0.9*x};
            \addplot [
                blue,
                area legend,
                opacity=0.2,
                ] fill between [
                    of=A and B
                ];
            \addplot [
                transparent,
                no markers,
                name path=C,
                domain=0:1e8
            ]{1.2   *x};
            \addplot [
                transparent,
                no markers,
                name path=D,
                domain=0:1e8
            ]{0.8*x};
            \addplot [
                blue,
                area legend,
                opacity=0.1,
                ] fill between [
                    of=C and D
                ];
            \legend{
                Предложенный алгоритм,
                Наивный подход,
                ,
                ,
                Область ошибки $10\%$,
                ,
                ,
                Область ошибки $20\%$,
            };
            \end{axis}
        \end{tikzpicture}
        \caption{Диаграмма рассеяния прогноза}
        \end{minipage} %    
        \begin{minipage}{0.45\textwidth}
        \centering
            \begin{tikzpicture}[font=\footnotesize]
                \begin{axis}[
                    ybar,
                    width=\linewidth,
                    ylabel=Абсолютная ошибка,
                    legend style={at={(0.5,1.2)},anchor=south},
                    legend columns=2]
                \addplot+[
                    hist={
                        bins=11,
                        data min=-3e6,
                        data max=3e6
                    },
                    opacity=0.5   
                ] table [y index=0] {\algoforecasterroroneweek};
                \addplot+[
                    hist={
                        bins=11,
                        data min=-3e6,
                        data max=3e6
                    },
                    opacity=0.5   
                ] table [y index=0] {\naiveforecasterroroneweek};
                \legend{Предложенный алгоритм, Наивный подход}
                \end{axis}
            \end{tikzpicture}
            \caption{Абсолютной ошибка}
        \end{minipage}
\end{figure}
\newpage
\subsubsection{Прогнозирование на 14 дней}
\begin{figure}[ht]
    \centering
    \begin{minipage}{0.45\textwidth}    
        \centering
        \begin{tikzpicture}[font=\scriptsize]
            \begin{axis}[
                unit vector ratio*=1 1 1,
                width=\linewidth,
                xlabel=Предсказанное значение,
                ylabel=Наблюдаемое значение,
                legend style={
                    at={(0.5,1.2)},
                    anchor=south,
                    nodes={right},
                    font=\scriptsize\mystrut},
                legend columns=2,
                xmax=1e7,
                ymin=-0.01,
                xmin=-0.01,
                ymax=1e7]
            \addplot+[
                only marks,
                mark=*,
                opacity=0.3]
            table {\algoforecasttwoweek};
            \addplot+[
                only marks,
                mark=*,
                opacity=0.3]
            table {\naiveforecasttwoweek};
            \addplot[
                transparent,
                no markers,
                name path=A,
                domain=0:1e8
            ]{1.1*x};
            \addplot[
                transparent,
                no markers,
                name path=B,
                domain=0:1e8
            ]{0.9*x};
            \addplot [
                blue,
                area legend,
                opacity=0.2,
                ] fill between [
                    of=A and B
                ];
            \addplot [
                transparent,
                no markers,
                name path=C,
                domain=0:1e8
            ]{1.2   *x};
            \addplot [
                transparent,
                no markers,
                name path=D,
                domain=0:1e8
            ]{0.8*x};
            \addplot [
                blue,
                area legend,
                opacity=0.1,
                ] fill between [
                    of=C and D
                ];
            \legend{
                Предложенный алгоритм,
                Наивный подход,
                ,
                ,
                Область ошибки $10\%$,
                ,
                ,
                Область ошибки $20\%$,
            };
            \end{axis}
        \end{tikzpicture}
        \caption{Диаграмма рассеяния прогноза}
        \end{minipage} %    
        \begin{minipage}{0.45\textwidth}
        \centering
            \begin{tikzpicture}[font=\footnotesize]
                \begin{axis}[
                    ybar,
                    width=\linewidth,
                    ylabel=Абсолютная ошибка,
                    legend style={at={(0.5,1.2)},anchor=south},
                    legend columns=2]
                \addplot+[
                    hist={
                        bins=11,
                        data min=-2e6,
                        data max=2e6
                    },
                    opacity=0.5   
                ] table [y index=0] {\algoforecasterrortwoweek};
                \addplot+[
                    hist={
                        bins=11,
                        data min=-2e6,
                        data max=2e6
                    },
                    opacity=0.5   
                ] table [y index=0] {\naiveforecasterrortwoweek};
                \legend{Предложенный алгоритм, Наивный подход}
                \end{axis}
            \end{tikzpicture}
            \caption{Абсолютной ошибка}
        \end{minipage}
\end{figure}