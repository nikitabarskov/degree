\section{Описание задачи}

Целью данного исследования является построения алгоритма, позволявщего с достаточной точностью качественно
предсказывать взаимодействие пользователя с рекламными элементами интернет страниц.
Следом данных взаимодействий является собраная рекламным сервером информация, представляющая собой набор
технических метаданных, куки данные пользователя и данные, собранные при помощи сторонних интеграций.
Поля, описывающие взаимодействия, представлены в таблице~\ref{tab:feature-description}.

\tabulinesep = 7pt
\begin{longtabu} to \textwidth {|X|X|X|}
        \caption{Описание признаков взаимодействия}
        \label{tab:feature-description}
        \endfirsthead
        \endhead
        \rowfont[c]{\bfseries}
        \hline
        Название поля & Описание & Множество допустимых значений \\
        \hline
        Идентификатор пользователя
        & Уникальный идентификатор пользователя, присваиваемый ему рекламным сервером
        & Строка, генерируемая согласно внутренней логике рекламного сервера \\
        \hline
        Идентификатор рекламного элемента
        & Уникальный идентификатор рекламного взаимодействия, с которым пользователь взаимодействовал
        & Целое число, генерируемое согласно внутренней логике рекламного сервера \\
        \hline
        Время взаимодействия 
        & Время взаимодействия пользователя с рекламным элементом в формате Unix-time
        & Целое число \\
        \hline
        Идентификатор географического положения пользователя 
        & Уникальный идентификатор географического положения пользователя
        & Целое число, генерируемое согласно внутренней логике рекламного сервера \\
        \hline
        Идентификатор используемого обозревателя интернет страниц 
        & Уникальный идентификатор интернет обозревателя, при помощи которого пользователь совершил
        взаимодействие с рекламным элементом
        & Целое число, генерируемое согласно внутренней логике рекламного сервера \\
        \hline
        Идентификатор операционной системы
        & Уникальный идентификатор операционной системы устройства, при помощи которого пользователь совершил
        взаимодействие с рекламным элементом
        & Целое число, генерируемое согласно внутренней логике рекламного сервера \\
        \hline
\end{longtabu}

Каждое взаимодействие можно представить в виде вектора $\mathbf{U_i}$, элементами которого являются признаки
взаимодействи.
\begin{equation}
    \mathbf{U_i}\left(t_k\right) =
    \left( u_1, u_2, \dots, u_n \right),
    i = \overline{1, M}, n = \overline{1, N},
\end{equation} где $M$~-- общее число взаимодействий, $N$~-- число признаков.

Один из признаков взаимодействия является поле, соответствующее временному отсчету взаимодействия $t$. В таком случае
исходные данные представляют собой многомерный временной ряд, в котором различные события (например взаимодействие
разных пользователей с одним и тем же рекламным элементом могут происходить одновременно).

\begin{equation}
    \mathbf{U_i}\left(t_k\right) =
    \left( u_1, u_2, \dots, u_n, t_k \right),
    i = \overline{1, M}, n = \overline{1, N}, k = \overline{0, T},
\end{equation} где $M$~-- общее число взаимодействий, $N$~-- число признаков, $T$~-- число временных отсчетов.

Задача данного исследования, на основе известных взаимодействий пользователей (множестве наблюдений)
\begin{equation}
    \left\{ \mathbf{U_i}\left(t_k\right)\right\},
    \forall i \in \left( 0, M \right), \forall k \in \left(0, T\right),
\end{equation}
предсказать взаимодействие пользователей на горизонте времени $\widehat{T}$
\begin{equation}
    \left\{\mathbf{\widehat{U}_i}\left(t_k\right)\right\},
    \forall i \in \left( M + 1, \widehat{M} \right), \forall k \in \left(T, \widehat{T}\right)ю
\end{equation}