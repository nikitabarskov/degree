\section{Описание алгоритма}
    
    Основная идея авторов работы заключается в комбинации существующего метода прогнозирования
    Naive Seasonal Approach (наивное сезонное прогнозирование) и метода калибровки выборки.

\subsection{Метрики, описывающие взаимодействия}

    Для определения способа калибровки наивного прогноза авторами рассмотрено поведение пользователей интернет сети.
    Особенностью этого поведения является возможное условное деление на 2 группы: первая из них является <<постоянным>>
    пользователем интернет ресурса (т.е. пользователь взаимодействовал с элементами рекламы несколько раз за выбранный
    период), другая же группа являются <<уходящими>> пользователям, т.е. взаимодействующими единожды.

    \paragraph{Постоянные и уходящие пользователи.} Предположим, мы рассматриваем взаимодействия пользователей с
    рекламным элементом $A$ на отрезке времени $\left[ T_0, T_1 \right]$ и $u_i$ -- признак взаимодействия,
    который показывает какой пользователь его совершил, $u_j$ -- признак взаимодействия, который показывает на какой
    интернет странице оно произошло, тогда если мы введем функцию 
    $\text{Imps}\left(U, A, \left\{\mathbf{U}\left(t_k\right)\right\}\right)$, 
    которая будет возвращать количество взаимодействий для выбранного пользователя $U$, то пользователь $U$ 
    будут постоянно взаимодействующим с рекламным элементом $A$, если
    \begin{equation}
        \text{Imps}\left(U, A, \left\{\mathbf{U}\left(t_k\right)\right\}\right) > 1, t_k \in \left[ T_0, T_1 \right]
    \end{equation}
    а уходящим на рекламном элементе $A$, если
    \begin{equation}
        \text{Imps}\left(U, A, \mathbf{U}\left(t_k\right)\right) \leq 1, t_k \in \left[ T_0, T_1 \right]
    \end{equation}

    Данное поведение, возможно описать двумя метриками: метрикой количества взаимодействий и
    метрикой количества уникальных взаимодействий.

    \paragraph{Метрика количества взаимодействий.} 
    Пусть дана выборка взаимодействий пользователей
    $\left\{\mathbf{U}\left(t_k\right)\right\}$ за период времени $\left[T_0, T_1\right]$, $u_i$ -- признак взаимодействия,
    который показывает какой пользователь его совершил, $u_j$ -- признак взаимодействия, который показывает на какой
    интернет странице оно произошло, тогда  $\text{Imps} \left( U, A, \left\{\mathbf{U}\left(t_k\right)\right\} \right)$ 
    -- взаимодействие пользователя $U$ c рекламным элементом $A$ в момент времени $t_k$.

    Также данная функция представима в следующем виде:

    \begin{equation}
        \text{Imps} \left( U, A, \left\{\mathbf{U}\left(t_k\right)\right\} \right) =
        \begin{cases}
            1, u_i = U \wedge u_j = A  \\
            0, u_i \neq U \vee u_j \neq A
        \end{cases}, \forall t_k \in \left[T_0, T_1\right]
    \end{equation}

    Метрика количества взаимодействий для рекламного элемента $A$ и пользователя $U$ за период $[T_0, T_1]$ 
    определяется следующим образом
    \begin{equation}
        \text{Imps} \left( U, A, \left\{\mathbf{U}\right\} \right) =
        \sum \limits_{t=T_0}^{T_1} \text{Imps} \left( U, A, U_i (t) \right) 
    \end{equation}


    \paragraph{Метрика количества уникальных пользователей.} Пусть дана выборка взаимодействий
    пользователей $\left\{\mathbf{U}\left(t_k\right)\right\}$ за период времени $[T_0, T_1]$,  $u_i$ -- признак 
    взаимодействия, который показывает какой пользователь его совершил, $u_j$ -- признак взаимодействия, 
    который показывает на какой интернет странице оно произошло, тогда
    $\text{Distinct}\left( A, \left\{\mathbf{U}\left(t_k\right)\right\} \right)$ -- количество 
    уникальных пользователей, взаимодействующих с рекламным элементом $A$ за период времени $[T_0, T_1]$.

    \begin{equation}
        \text{Distinct}\left( A, \left\{\mathbf{U}\left(t_k\right)\right\} \right) =
        \begin{cases}
            \left|\left\{u_i\right\}\right|, u_j = A  \\
            0, u_j \neq A
        \end{cases}, \forall t_k \in \left[T_0, T_1\right]
    \end{equation}

    Метрика количества уникальных пользователей не является аддитивной как метрика количества взаимодействий,
    так как основана на мощности множества, состоящего из элементов $u_i$ (множества уникальных 
    идентификаторов пользователей). 

\subsection{Построение прогнозов метрик}

На основе исходных наблюдений можно построить временные прогнозы метрик количества наблюдений и
количества уникальных пользователей.

Для построения одномерного прогноза необходимо:
\begin{enumerate}
    \item Из исходного ряда наблюдений построить одномерные временные ряды для метрик количества
    наблюдений
    \begin{equation}
        \text{Imps} \left(A_i, \left\{\mathbf{U}\left(t_k\right)\right\} \right) 
        = \sum \limits_{\forall U_j} \text{Imps} \left( U_j, A_i, \left\{\mathbf{U}\left(t_k\right)\right\} \right), 
        \forall t_k \in \left[T_0, T_1\right]
    \end{equation}

    Если число уникальных рекламных элементов известно и допустим оно равно $N_A$, то после проведения данной операции
    мы получаем $N_A$ наборов одномерных временных рядов метрики количества взаимодействий. Каждая из них будет
    соответствовать одному рекламному элементу.

    \item Из исходного ряда наблюдений построить одномерные временные ряды для метрик количества
    наблюдений
    \begin{equation}
        \text{Distinct}\left( A_i, \left\{\mathbf{U}\left(t_k\right)\right\} \right), i = \overline{0, N_A - 1}, \forall t_k \in \left[T_0, T_1\right]
    \end{equation}

    \item Построить временные прогнозы каждой из метрика каждого рекламного элемента $A_i$.
\end{enumerate}

\subsection{Построение многомерного временного прогноза}

    При построении многомерного временного прогноза используется метод наивного сезонного прогнозирования. 
    Если период сезонности $\nu$, то прогноз с использованием статистики за периода $\left[T_0, T_1\right]$ 
    можно представить в следующем виде
    \begin{equation}
        \widehat{\mathbf{U}}\left(t_k + h\right) = \mathbf{U}\left(t_k + h - \nu \right), t_k \in 
    \end{equation} где $h$~-- горизонт прогнозирования, $\nu$~-- период сезонности.

    \subsection{Построение калибровочной функции}

    Построенные прогнозы многомерного временного ряда и метрик позволяют осуществить операцию калибровки
    наивного мномерного прогноза к одномерным прогнозам метрик. Для осуществления данной операции необходимо
    построить набор метрик количества взаимодействий и количества уникальных взаимодействий пользователей на
    всех рекламных элементах за период $\left[T_1, T_2\right]$.

    Так для каждого пользователя $U_i$ получим метрики взаимодействией с каждым рекламным элементом $A_j$
    \begin{equation}
        \text{Imps}\left(U_i, A_j, \left\{\widehat{\mathbf{U}}\left(t_k\right)\right\}\right)
    \end{equation}

    Если взять идентификатор рекламного элемента и временной индекс $(A_j, t_k)$ в качестве строчного индекса
    матрицы, а идентификатор пользователя в качестве колоночного индекса матрицы, то мы получим матрицу взаимодействий
    \begin{equation}
        \text{MIMPS} = a_{ml},
    \end{equation} где $m \in {A_j \times t_k}$~-- набор пар идентификатора рекламного элемента и временного отсчета, 
    $l \in {U_i}$~-- идентификатор пользователя. 

    Элементы данной матрицы отражают число взаимодействий пользователя $U_i$ с рекламным элементом $A_j$ в момент времени $t_k$.

    Аналогичную операцию можно провести с набором прогнозом метрик. В результате мы получим вектор, индексом которого будет
    пара идентификатора рекламного элемента и временного отсчета.
    \begin{equation}
        \text{IMPS} = f_{m},
    \end{equation} где $m \in {A_j \times t_k}$~-- набор пар идентификатора рекламного элемента и временного отсчета. 
    Элементы данного вектора будут отражать количество взаимодействий, которые совершили пользователи с рекламным 
    элементом $A_j$ в момент времени $t_k$.\
    

    Задача калибровки в данном случае -- при помощи решения СЛАУ составленной из матрицы взаимодействий пользователей
    $MIPMS$ и вектора количества взаимодействий $IMPS$ определить взвешенное число взаимодействий пользователя 
    в будущем.

    \begin{equation}
        \left(
            \begin{array}{ccc}
                a_{00} & \dots & a_{0L}  \\
                \vdots & \ddots & \vdots\\
                a_{M0} & \dots & a_{ML}
            \end{array}
        \right) \times
        \left(
            \begin{array}{ccc}
                us_0 \\
                \vdots \\
                us_L
            \end{array}
        \right) = 
        \left(
            \begin{array}{ccc}
                f_0 \\
                \vdots \\
                f_m
            \end{array}
        \right)
    \end{equation}

    Вектор решения $US$~-- является калибровочным вектором, который показывает какое число взаимодействий будущем
    пользователь совершит.

