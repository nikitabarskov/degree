\documentclass[a4paper, 12pt]{extreport}

\usepackage{extsizes}
\usepackage[left=3cm,right=1.5cm,top=1.25cm,bottom=2cm]{geometry}

\usepackage[english,russian]{babel}
\usepackage[T1]{fontenc}
\usepackage[utf8]{inputenc}
\usepackage{fontspec}

\usepackage{ragged2e}
\usepackage{microtype}

\justifying
\sloppy
\tolerance=1000
\hyphenpenalty=10000
\emergencystretch=3em

\usepackage{calc}

\setmainfont{Times}
\setsansfont{Helvetica Neue}
\setmonofont{Fira Code}

\usepackage{graphicx}

\usepackage[tableposition=top]{caption}
\usepackage{subcaption}
\DeclareCaptionLabelFormat{gostfigure}{Рисунок #2}
\DeclareCaptionLabelFormat{gosttable}{Таблица #2}
\DeclareCaptionLabelSeparator{gost}{~---~}
\captionsetup{labelsep=gost}
\captionsetup[figure]{labelformat=gostfigure}
\captionsetup[table]{labelformat=gosttable}
\renewcommand{\thesubfigure}{\asbuk{subfigure}}

\usepackage{amsmath}
\usepackage{tabu}
\usepackage{longtable}
\pagestyle{empty}
% Document
\begin{document}
    \begin{center}
        МИНИСТЕРСТВО НАУКИ И ВЫСШЕГО ОБРАЗОВАНИЯ РОССИЙСКОЙ ФЕДЕРАЦИИ
        \vspace{12pt}

        федеральное государственное автономное образовательное учреждение высшего \\ образования
        <<Самарский национальный исследовательский университет имени академика \\ С.П. Королева>> \\
        (Самарский университет)
        \vspace{12pt}

        Институт информатики, математики и электроники \\
        Факультет информатики \\
        Кафедра информационных систем и технологий
        \vspace{60pt}

        \textbf{ОТЧЕТ ПО ПРАКТИКЕ}
        \vspace{36pt}

        Вид практики \underline{производственная} \\
        (учебная, производственная)
        \vspace{12pt}

        Тип практики \underline{практика по получению профессиональных умений и опыта} \\
        \underline{профессиональной деятельности} \\
        (в соответствии с ОПОП ВО)
        \vspace{12pt}

        Сроки прохождения практики: с 01.09.2018 по 29.12.2018 \\
        (в соответствии с календарным учебным графиком)
        \vspace{12pt}

        по направлению подготовки 09.04.01 Информатика и вычислительная техника \\
        (уровень магистратуры) \\
        направленность (профиль) «Автоматизированные системы обработки информации \\ и управления»
    \end{center}

    \noindent Студент группы № 6223-090401D \hrulefill \mbox{ Н.М. Барсков}

    \noindent Руководитель практики, \hfill \break
    доцент кафедры ИСТ, к.т.н. \hrulefill \mbox{ И.М. Куликовских}\vspace{36pt}

    \noindent Дата сдачи 29.12.2018 \\
    \noindent Дата защиты 29.12.2018
    \vspace{12pt}

    \noindent Оценка \rule{\widthof{неудовлетворительно}}{0.4pt}
    \vfill

    \begin{center}
        \noindent Самара 2018
    \end{center}
    \newpage
    \tableofcontents
    \newpage
    \begin{center}
        МИНИСТЕРСТВО НАУКИ И ВЫСШЕГО ОБРАЗОВАНИЯ РОССИЙСКОЙ ФЕДЕРАЦИИ
        \vspace{14pt}

        федеральное государственное автономное образовательное учреждение высшего \\ образования
        <<Самарский национальный исследовательский университет имени академика \\ С.П. Королева>> \\
        (Самарский университет)
        \vspace{14pt}

        Институт информатики, математики и электроники \\
        Факультет информатики \\
        Кафедра информационных систем и технологий
        \vspace{70pt}

        \addcontentsline{toc}{chapter}{Индивидуальное задание на практику}
        \textbf{Индивидуальное задание на практику}
        \vspace{14pt}
    \end{center}

    \noindent Студенту Барскову Н.М. группы 6223-090401D \\
    Направление на практику оформлено приказом по университету от
    \rule{\widthof{99}}{0.4pt}.\rule{\widthof{99}}{0.4pt}.20\rule{\widthof{99}}{0.4pt} г.
    №\rule{\widthof{9999999}}{0.4pt} на кафедру информационных систем и технологий

    \noindent\begin{longtable}{|m{4.5cm}|m{6.5cm}|m{4.5cm}|}
                 \hline
                 Планируемые результаты освоения образовательной программы (компетенции) &
                 Планируемые результаты практики &
                 {Содержание задания} \\
                 \hline
                 ПК-10 способностью разрабатывать и реализовывать планы информатизации предприятий и их подразделений
                 на основе Web- и CALS-технологий &
                 Знать: современные направления развития интеллектуальных систем и программные реализации информационных
                 интеллектуальных технологий, в том числе положения Web- и CALS-технологий; сферы, аспекты применения,
                 возможности и концепции информатизации предприятий и их структурных подразделений на основе Web- и
                 CALS-технологий
                 & Собрать и проанализировать данные о взаимодействиях пользователей с рекламными элементами сайта \\

                 & Уметь: формировать требования к интеллектуальной системе и определять возможные пути их выполнения;
                 разрабатывать концепцию информатизации предприятий в соответствии с их функциональными потребностями;
                 выбирать инструментальные средства и технологии создания интеллектуальных информационных систем
                 предприятия в соответствии со структурой его системы управления & \\
                 \hline

                 & Владеть: навыками работы с инструментальными средствами моделирования предметной области и прикладных
                 процессов; навыками разработки технологической документации; использования функциональных и
                 технологических стандартов интеллектуальных информационных систем &  \\
                 \hline
                 ПК-11 способностью формировать технические задания и участвовать в разработке аппаратных и (или)
                 программных средств вычислительной техники
                 & Знать: полный цикл разработки программного обеспечения.
                 Уметь: формировать технические задания; участвовать в разработке аппаратных и (или) программных средств
                 (или) программных средств вычислительной техники вычислительной техники.
                 Владеть: навыками систематизации информации; навыками работы в команде.
                 & На основе анализа данных о пользователях сделать преположения по решению проблемы прогнозирования
                 данного взаимодействия \\
                 \hline
                 ПК-12 способностью выбирать методы и разрабатывать алгоритмы решения задач управления и проектирования
                 объектов автоматизации
                 & Знать: системообразующую роль ПО. Проблемы разработки ПО и пути их решения, тенденцию избыточного
                 распространения информации в программе и приемы сокрытия информации, виды контроля работы ПО, методы
                 контроля работы ПО встроенными средствами без прекращения его функционирования, методы обработки
                 возможных ошибок, обнаруженных во входных и выходных данных, три вида программных разработок с
                 точки зрения технологии их создания и эксплуатации, цену ошибок проектирования, закон Рамамуртию.
                 Уметь: использовать особенности коллективной разработки ПО СТС, защищаться от ошибок ПО, создавать
                 структуру ПО, временную диаграмму работы системы с параллельными физическими процессами, &
                 Изучив основые подходы к прогнозированию временных рядов, выбору исходных признаков для модели
                 прогнозирования и методы статистического анализа предложить решение по решению проблемы
                 прогнозирования взаимодействия пользователей с рекламным элементами интернет-страниц \\
                 \hline

                 & проектировать многозадачную работу ПО, реализовывать многозадачность за счет параллельных вычислений,
                 обосновывать и выбирать соответствующий вариант технологии разработки ПО, проводить минимизацию
                 сложности ПО, использовать стандартные приемы в конструировании. & \\
                 & Владеть: обменом информацией между взаимодействующими задачами, технологией Open МР, эвристическими
                 принципами конструирования программ, предотвращением дублирования кода, защитой от прерывания программы
                 в «неудобное» время, перечнем нештатных ситуаций и аварийной защитой, защитным конструированием ПО,
                 видами и структурой документов на ПО, выпускаемых по этапам разработки системы и ПО, внешними
                 (пользователя) и внутренними характеристиками качества ПО - характеристиками разработчика. & \\
                 \hline
    \end{longtable}

    \noindent Дата выдачи заявления 01.09.2018. \\
    Срок представления на кафедру отчета по практике 29.12.2018. \\
    \vspace{24pt}

    \noindent Руководитель практики, \\
    доцент кафедры ИСТ, к.т.н. \hrulefill\mbox{ И.М. Куликовских} \\
    \vspace{12pt}

    \noindent Задание к исполнению принял \\
    студент группы 6223-090401D \hrulefill\mbox{ Н.М. Барсков}
    \newpage
    \addcontentsline{toc}{chapter}{Рабочий график (план) проведения практики}
    \begin{center}
        \textbf{Рабочий график (план) проведения практики}
    \end{center}
    \noindent\begin{longtable}{|m{4.5cm}|m{6.5cm}|m{4.5cm}|}
                 \hline
                 Дата(период) &
                 Содержание задания &
                 Результаты практики \\
                 \hline
                 01.09.2018
                 & Собрать и проанализировать данные о взаимодействиях пользователей с рекламными элементами сайта
                 & Выполнено \\
                 \hline
                 01.11.2018
                 & На основе анализа данных о пользователях сделать преположения по решению проблемы прогнозирования
                 данного взаимодействия
                 & Выполнено \\
                 \hline
                 29.12.2019
                 & Изучив основые подходы к прогнозированию временных рядов, выбору исходных признаков для модели
                 прогнозирования и методы статистического анализа предложить решение по решению проблемы
                 прогнозирования взаимодействия пользователей с рекламным элементами интернет-страниц
                 & Выполнено \\
                 \hline
    \end{longtable}

    \noindent Руководитель практики, \\
    доцент кафедры ИСТ, к.т.н. \hrulefill \mbox{И.М. Куликовских}

    \newpage
    \begin{center}
        \textbf{Описательная часть}
    \end{center}
    \addcontentsline{toc}{chapter}{Описательная часть}

    Механизм сбора данных о взаимодействии пользователя с рекламными элементами интернет-страниц заключается в
    формировании профиля пользователя, который включает в себя время взаимодействия, информацию о рекламном элементе,
    \textit{cookie} данные (техническая информация об устройстве, с которого было произведено взаимодействие,
    программное обеспечение, при помощи которого было осуществлено взаимодействие, служебные данные системы,
    осуществляющей сбор сведений о взаимодействии), данные, полученные из внешних источников.

    Профиль пользователя представляет собой многомерный временной ряд, описывающий процесс взаимодействия пользователей
    с рекламными элементами сайта.

    В рамках предметной области для качественного описания взаимодействия используются метрики количества
    пользователей и количества уникальных пользователей с заданным параметрами профиля.

    В рамках данной производственной практики были проанализированы структура и содержание обезличенных данных
    о взаимодействиях пользователей с рекламным содержимым интернет-ресурса.

    Данные представляют собой бинарные файлы в формате Apache Parquet и содержат информацию о взаимодействии
    пользователей с 1 марта 2018 года по настоящее время.

    Для построения модели и решения доступны следующие поля:
    \textit{уникальный идентификационный номер пользователя},
    \textit{уникальный идентификационный номер рекламного элемента},
    \textit{время взаимодействия}, \textit{тип взаимодействия},
    \textit{cookie}.

    Для первичного анализа и описания решения, были построены временные ряды метрик количества взаимодействий и
    количества взаимодействий уникальных пользователей.

    \begin{figure}[ht]
        \centering
        \includegraphics[scale=0.15]{img/time-series-uniques-2018-10-01-2018-12-31.png}
        \caption{Примеры анализируемых временных рядов (количество уникальных взаимодействий)}
    \end{figure}

    \begin{figure}[ht]
        \centering
        \includegraphics[scale=0.15]{img/time-series-impressions-2018-10-01-2018-12-31.png}
        \caption{Примеры анализируемых временных рядов (количество взаимодействий)}
    \end{figure}

    В качественной решения задачи прогнозирования взаимодействий пользователей с рекламным содержимым интернет-страниц
    предлагается комбинация статистичеких и алгебраических методов. Основная идея заключается в калибровке существующей
    выборке пользователей к прогнозируемым значениям метрик количества взаимодействий и количества уникальных
    взаимодействий.

    При данной выборке $X$, описывающей поведение $N$ пользователей, производится построение прогноза количества
    взаимодействий $Y$ и количества уникальных взаимодействий $Z$.

    Затем, при помощи оптимизационного алгоритма производится подбор калибровочных коэффициентов $W$ так, чтобы
    метрики по исходной выборке соответствовали прогнозируемым метрикам с заданной точностью $\varepsilon$.

    Для проверки данной гипотезы, был проведен экперимент, реализующий прогноз взаимодействий трехсот тысяч пользователей
    с полутора тысячами рекламными элементами интернет-страниц.

    \begin{figure}[ht]
        \centering
        \begin{subfigure}[b]{0.4\textwidth}
            \centering
            \includegraphics[scale=0.2]{img/scatter-plot-impression.png}
            \caption{}
        \end{subfigure} %
        \begin{subfigure}[b]{0.4\textwidth}
            \centering
            \includegraphics[scale=0.2]{img/histogram-impressions.png}
            \caption{}
        \end{subfigure}
        \caption{(a) Оценка прогноза (диаграмма рассеяния, количество взаимодействий);
        (б) Оценка прогноза (распределение средней абсолютной взвешенной ошибки, количество взаимодействий)}
    \end{figure}

    \begin{figure}[ht]
        \centering
        \begin{subfigure}[b]{0.4\textwidth}
            \centering
            \includegraphics[scale=0.2]{img/scatter-plot-uniques.png}
            \caption{}
        \end{subfigure} %
        \begin{subfigure}[b]{0.4\textwidth}
            \centering
            \includegraphics[scale=0.2]{img/histogram-uniques.png}
            \caption{}
        \end{subfigure}
        \caption{(a) Оценка прогноза (диаграмма рассеяния, количество уникальных взаимодействий);
        (б) Оценка прогноза (распределение средней абсолютной взвешенной ошибки, количество уникальных взаимодействий)}
    \end{figure}

    Для реализации данного решения предлагается использовать Apache Spark для обработки больших данных, Apache Hadoop
    для хранения данных, для статистического анализа и получения метрик Apache Spark и его интерфейс для языка
    программирования Python 3 PySpark, библиотеку Pandas, библиотеку прогнозирования временных рядов Facebook Prophet,
    библиотеку SciPy и Scikit-learn для реализации оптимизационных методов.

    В рамках данной работы была проведена проверка концепции предложенного метода, реализована программная версия
    данной концепции, проведены сравнительные эксперименты для оценки качества данной модели.
\end{document}