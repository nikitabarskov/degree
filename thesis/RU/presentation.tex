\documentclass[handout]{beamer}
\title{Прогнозирование взаимодействий пользователей с рекламными элементами интернет-страниц}
\author{Студент Барсков Никита Максимович, гр. 6223-09.04.01D \\ Руководитель: Куликовских И.М., к.т.н., доцент кафедры ИСТ}
\titlegraphic{\centering\includegraphics[width=0.6\textwidth]{inc/images/su.png}}
\date{Самара, 2019}

\usepackage{fontspec}
\usetheme[block=fill]{metropolis}
\setsansfont{Helvetica Neue}
\setmonofont{Fira Code}

\usepackage{pgfplotstable}
\usepgfplotslibrary{dateplot}
\usepgfplotslibrary{fillbetween}
\usepgfplotslibrary{statistics}

\usepackage{tikz}
\usetikzlibrary{shapes.geometric}
\usetikzlibrary{arrows}
\usetikzlibrary{fit}
\usetikzlibrary{positioning}
\usetikzlibrary{calc}
\usetikzlibrary{shapes.misc}

\usepackage{adjustbox}

\pgfplotstableread[
    col sep=comma
    ]{inc/csv/network_1_ts.csv} 
    \networktimeseries{}

\pgfplotstableread[
    col sep=comma
    ]{inc/csv/network_1_ts_decompose.csv} 
    \timeseriesdecompose{}

\pgfplotstableread[
    col sep=comma
    ]{inc/csv/uniques-10-days.csv} 
    \uniquescumsumtendays{}

\pgfplotstableread[
    col sep=comma
    ]{inc/csv/fourier.csv} 
    \fourierdataset{}

\pgfplotstableread[
    col sep=comma
    ]{inc/csv/network.csv} 
    \networkimpressions{}

\pgfplotstableread[
        col sep = comma
        ]{inc/csv/eval-2019-02-28-1day.csv}
        \fbteod{}

\pgfplotstableread[
        col sep = comma
        ]{inc/csv/mae-2019-02-28-1day-naive.csv}
        \fbteodmaenaive{}

\pgfplotstableread[
        col sep = comma
        ]{inc/csv/mae-2019-02-28-1day-calibrated.csv}
        \fbteodmaecalibrated{}

\pgfplotstableread[
        col sep = comma
        ]{inc/csv/mape-2019-02-28-1day-naive.csv}
        \fbteodmapenaive{}        

\pgfplotstableread[
        col sep = comma
        ]{inc/csv/mape-2019-02-28-1day-calibrated.csv}
        \fbteodmapecalibrated{}

\pgfplotstableread[
        col sep = comma
        ]{inc/csv/eval-2019-02-28-7days.csv}
        \fbtesd{}

\pgfplotstableread[
        col sep = comma
        ]{inc/csv/mae-2019-02-28-7days-naive.csv}
        \fbtesdmaenaive{}

\pgfplotstableread[
        col sep = comma
        ]{inc/csv/mae-2019-02-28-7days-calibrated.csv}
        \fbtesdmaecalibrated{}

\pgfplotstableread[
        col sep = comma
        ]{inc/csv/mape-2019-02-28-7days-naive.csv}
        \fbtesdmapenaive{}        

\pgfplotstableread[
        col sep = comma
        ]{inc/csv/mape-2019-02-28-7days-calibrated.csv}
        \fbtesdmapecalibrated{}

\pgfplotstableread[
            col sep = comma
            ]{inc/csv/eval-2019-02-28-3days.csv}
            \fbtefd{}

\pgfplotstableread[
        col sep = comma
        ]{inc/csv/mae-2019-02-28-3days-naive.csv}
        \fbtefdmaenaive{}

\pgfplotstableread[
        col sep = comma
        ]{inc/csv/mae-2019-02-28-3days-calibrated.csv}
        \fbtefdmaecalibrated{}

\pgfplotstableread[
        col sep = comma
        ]{inc/csv/mape-2019-02-28-3days-naive.csv}
        \fbtefdmapenaive{}

\pgfplotstableread[
        col sep = comma
        ]{inc/csv/mape-2019-02-28-3days-calibrated.csv}
        \fbtefdmapecalibrated{}

    \tikzstyle{entity} = [
    rectangle,  
    minimum width=3cm, 
    minimum height=1cm,
    font=\small,
    text centered, 
    text width=2.5cm,
    draw=black,
    fill=cyan!20]

\tikzstyle{arrow} = [
    thick,
    ->]

    \makeatletter
    \pgfdeclareshape{datastore}{
      \inheritsavedanchors[from=rectangle]
      \inheritanchorborder[from=rectangle]
      \inheritanchor[from=rectangle]{center}
      \inheritanchor[from=rectangle]{base}
      \inheritanchor[from=rectangle]{north}
      \inheritanchor[from=rectangle]{north east}
      \inheritanchor[from=rectangle]{east}
      \inheritanchor[from=rectangle]{south east}
      \inheritanchor[from=rectangle]{south}
      \inheritanchor[from=rectangle]{south west}
      \inheritanchor[from=rectangle]{west}
      \inheritanchor[from=rectangle]{north west}
      \backgroundpath{
        \southwest \pgf@xa=\pgf@x \pgf@ya=\pgf@y
        \northeast \pgf@xb=\pgf@x \pgf@yb=\pgf@y
        \pgfpathmoveto{\pgfpoint{\pgf@xa}{\pgf@ya}}
        \pgfpathlineto{\pgfpoint{\pgf@xb}{\pgf@ya}}
        \pgfpathmoveto{\pgfpoint{\pgf@xa}{\pgf@yb}}
        \pgfpathlineto{\pgfpoint{\pgf@xb}{\pgf@yb}}
     }
    }
    \makeatother
    \tikzstyle{source} = [
        draw,
        thick,
        rounded corners,
        fill=yellow!20,
        minimum width=2cm, 
        minimum height=1cm,
        text centered, 
        text width=2cm]
    \tikzstyle{process} = [
        draw,
        thick,
        circle,
        minimum size=2.5cm,
        inner sep=1pt,
        fill=blue!20,
        text centered, 
        text width=2cm]
    
    \tikzstyle{datastore} = [
        draw,
        very thick,
        shape=datastore,
        inner sep=1pt,
        minimum width=2.5cm, 
        minimum height=1cm,
        text centered, 
        text width=2cm]
\begin{document}
\maketitle

\begin{frame}
    \frametitle{Цели работы}
    Разработка алгоритма прогнозирования взаимодействий пользователей с рекламным содержимым интернет-страниц и оценка качества работы разработанного решения.\\
\end{frame}

\begin{frame}
    \frametitle{Задачи работы}
    \begin{itemize}
        \item анализ процесса взаимодействия пользователей с рекламными элементами интернет-страниц; 
        \item математическая постановка задачи прогнозирования; 
        \item разработка алгоритма прогнозирования взаимодействий пользователей; 
        \item программная реализация разработанного алгоритма; 
        \item оценка качества работы разработанного алгоритма в сравнении с существующим методом наивного сезонного прогнозирования. 
    \end{itemize}
\end{frame}

\begin{frame}
    \frametitle{Актуальность работы}
    \begin{itemize}
        \item активное развитие индустрии онлайн рекламы требует появления новых алгоритмов, способствующих эволюции области;
        \item применение современных математических моделей для решение поставленной задачи;
        \item использование релевантных средств разработки алгоритмов и систем обработки данных.
    \end{itemize}
\end{frame}

\begin{frame}
    \frametitle{Схема реализации рекламного содержимого}

\begin{figure}[h!]
        \centering
        \resizebox{\columnwidth}{!}{%
        \begin{tikzpicture}
            \node (user_1) [entity] {Посетитель 1};
            \node (user_2) [entity, below of=user_1, yshift=-0.5cm] {Посетитель 2};
            \node (user_3) [entity, below of=user_2, yshift=-0.5cm] {Посетитель 3};
            \node (page) [entity, right of=user_2, xshift=2.5cm] {Интернет страница с рекламным элементом};
            \draw [arrow] (user_1) -| node[anchor=south, font=\small, text width=3cm] {Уведомление о показе рекламного места} (page);
            \draw [arrow] (user_2) -- (page);
            \draw [arrow] (user_3) -| (page);
    
            \node (ssp) [entity, right of=page, xshift=2.5cm] {SSP};
                
            \draw [arrow] (page) -- (ssp);
                
            \node (dsp_2) [entity, right of=ssp, xshift=2.5cm] {DSP 2};
            \node (dsp_1) [entity, above of=dsp_2, yshift=0.5cm] {DSP 1};
            \node (dsp_3) [entity, below of=dsp_2, yshift=-0.5cm] {DSP 3};
    
            \draw [arrow] (ssp) |- node[anchor=south, font=\small, text width=3cm] {Предоставление рекламного места для продажи} (dsp_1);
            \draw [arrow] (ssp) -- (dsp_2);
            \draw [arrow] (ssp) |- (dsp_3);
    
            \node (advertiser_1) [entity, right of=dsp_1, xshift=2.5cm, yshift=1cm] {Рекламодатель 1};
            \node (advertiser_2) [entity, right of=dsp_2, xshift=2.5cm] {Рекламодатель 2};
            \node (advertiser_3) [entity, right of=dsp_3, xshift=2.5cm, yshift=-1cm] {Рекламодатель 3};
    
            \draw [arrow] (dsp_1) |- node[anchor=south, font=\small, text width=3cm] {Запрос рекламного содержимого} (advertiser_1);
            \draw [arrow] (dsp_2) -- (advertiser_2);
            \draw [arrow] (dsp_3) |- (advertiser_3);
        \end{tikzpicture}
        }
\end{figure}
\end{frame}

\begin{frame}
    \frametitle{Протокол работы и анализ взаимодействий}
    \begin{block}{Протокол работы}
    Файл или совокупность файлов, в котором содержатся хронологический порядок
    взаимодействий пользователей с рекламными элементами интернет-страниц и описание данных взаимодействий.
    \end{block}
    \begin{block}{Число взаимодействий}
        Количество показов, переходов и иных взаимодействий пользователей с рекламным элементом.
        \end{block}
        \begin{block}{Число уникальных пользователей}
            Число неповторяющихся пользователей, осуществивших взаимодействие с конкретным рекламным элементом интернет-страницы за определенный период времени.
        \end{block}
\end{frame}

\begin{frame}
    \frametitle{Количественная оценка взаимодействий}
    Каждое взаимодействие можно описать в виде
    \begin{equation*}
        \mathbf{U_i} = \left(t_k, u_1, \dots, u_n \right),
        \; i = \overline{1, M},
        \; n = \overline{1, N},
        \; k = \overline{1, K},
        \; u_n \in D_n,
    \end{equation*}
    тогда функция взаимодействия пользователя с рекламным элементом выглядит так
    \begin{equation*}
        \text{Imps} \left( U, A, t_k, \left\{ \mathbf{U} \right\} \right) =
            \begin{cases}
                1,\; u_i = U \wedge u_j = A  \\
                0,\; u_i \neq U \vee u_j \neq A
            \end{cases},
            \; \forall t_k \in \left[ T_0, T_1 \right],
        \label{eq:user-imps-definition}
    \end{equation*}
    а метрика числа уникальных пользователей
    \begin{equation*}
        \text{Distinct}\left( A, T_0, T_1, \left\{\mathbf{U}\right\} \right) =
        \begin{cases}
            \left|\left\{u_i\right\}\right|,\; u_j = A  \\
            0,\; u_j \neq A
        \end{cases}, \forall t_k \in \left[T_0, T_1\right],
    \end{equation*}
\end{frame}

\begin{frame}
    \frametitle{Количественная оценка взаимодействий}
    \centering
    \begin{figure}[ht!]
        \resizebox{\columnwidth}{!}{%
        \begin{tikzpicture}[font=\footnotesize]
            \begin{axis}[
                /pgf/number format/.cd,
                    fixed,
                    use comma,
                /tikz/.cd,
                date coordinates in=x,
                width=\textwidth,
                height=.4\textwidth,
                xmin=2018-03-01,
                xmax=2019-03-01,
                xtick distance=90,
                tick scale binop=\times,
                xlabel=Дата,
                ylabel style={align=center},
                ylabel=Число\\взаимодействий]
                \addplot+[no markers] table[x=date, y=impressions] {\networktimeseries};
            \end{axis}
        \end{tikzpicture}
        }
    \end{figure}
\end{frame}

\begin{frame}
    \frametitle{Алгоритм прогнозирования}
    \tikzstyle{startstop} = [
    rounded rectangle,
    minimum width=4cm,
    minimum height=1cm,
    text centered,
    text width=3.75cm,
    draw=black, 
    fill=red!20]

\tikzstyle{io} = [
    trapezium,
    trapezium left angle=70, 
    trapezium right angle=110, 
    minimum width=4cm, 
    minimum height=1cm, 
    text centered, 
    text width=3.75cm,
    draw=black, 
    fill=blue!20]

\tikzstyle{process} = [
    rectangle, 
    minimum width=4cm,
    minimum height=1cm, 
    text centered,
    text width=3.75cm,
    draw=black, 
    fill=yellow!20]

\tikzstyle{arrow} = [
        thick,
        ->]
\begin{columns}
    \column{0.25\textwidth}
    \begin{figure}[h!]
        \centering
        \begin{adjustbox}{width=0.9\textwidth}
        \begin{tikzpicture}[font=\small]
            \node [startstop] (start) {Инициализация};
            \node [io, below of=start, yshift=-0.5cm] (logs) {Исходный протокол работы};
            \node [process, below of=logs, yshift=-1cm] (sample) {Построение вероятностной выборки};
            \node [process, below of=sample, yshift=-1cm] (reports) {Построение статистики метрик};
            \node [process, below of=reports, yshift=-1cm,] (naive) {Построение прогноза протокола работы};
            \node [process, below of=naive, yshift=-1cm] (forecast) {Построение прогноза метрик};
            \node [process, below of=forecast, yshift=-1cm] (calib) {Калибровочный процесс};
            \node [io, below of=calib, yshift=-1cm] (results) {Прогноз пользовательских взаимодействий};
            \node [startstop, below of=results, yshift=-1cm] (end) {Конец};
    
            \draw [arrow] (start) -- (logs);
            \draw [arrow] (logs) -- (sample);
            \draw [arrow] (sample) -- (reports);
            \draw [arrow] (reports) -- (naive);
            \draw [arrow] (naive) -- (forecast);
            \draw [arrow] (forecast) -- (calib);
            \draw [arrow] (calib) -- (results);
            \draw [arrow] (results) -- (end);
        \end{tikzpicture}
        \end{adjustbox}
    \end{figure}
    \column{0.75\textwidth}
    \begin{itemize}
        \item проведение вероятностной выборки взаимодействий и получение метрик; 
        \item построение прогноза протокола работы рекламной системы и построение прогноза метрик; 
        \item проведение калибровочного процесса для прогноза протокола работы рекламной системы и прогноза метрик.
    \end{itemize}
\end{columns}
\end{frame}

\begin{frame}
    \frametitle{Алгоритм прогнозирования}
    Для построения прогноза протокола работы воспользуемся методом наивного сезонного прогнозирования
    \begin{equation*}
        \hat{\mathbf{U}}_{\left.T+h\right|T} = \mathbf{U}_{T+h-(k+1)m}.
    \end{equation*}
    При построении прогнозов метрик воспользуемся определением прогнозирования одномерного временного ряда
    \begin{equation*}
        \hat{y}_{\left.T+h\right|T} = F\left(y_0, \dots, y_T\right),
    \end{equation*}
    и классической моделью декомпозиции для построения прогноза
    \begin{equation*}
        y(t) = g(t) + s(t) + h(t) + \varepsilon(t),
    \end{equation*}
\end{frame}

\begin{frame}
    \frametitle{Алгоритм прогнозирования}
    Калибровочный процесс представляет собой решение матрицы
    \begin{equation*}
        \left(
            \begin{array}{ccc}
                \text{MIMPS}_{00} & \dots & \text{MIMPS}_{0L} \\
                \vdots & \ddots & \vdots \\
                \text{MIMPS}_{M0} & \dots & \text{MIMPS}_{ML} \\
                \text{MDSTN}_{00} & \dots & \text{MDSTN}_{0L} \\
                \vdots & \ddots & \vdots \\
                \text{MDSTN}_{M0} & \dots & \text{MDSTN}_{ML} \\
            \end{array}
        \right)\times
        \left(
            \begin{array}{c}
                \text{us}_{0} \\
                \vdots \\
                \vdots \\
                \vdots \\
                \vdots \\
                \text{us}_{L}
            \end{array}
        \right) = 
        \left(
            \begin{array}{c}
                \text{IMPS}_0 \\
                \vdots \\
                \text{IMPS}_M \\
                \text{DSNT}_0 \\
                \vdots \\
                \text{DSNT}_M
            \end{array}
        \right).
    \end{equation*}
\end{frame}

\begin{frame}
    \frametitle{Реализация алгоритма}
    

\begin{figure}[ht!]
    \centering
    \resizebox{\columnwidth}{!}{%
    \begin{tikzpicture}[font=\small]
        \node (source_data) [source] {Протокол работы рекламной системы};
        \node (source_logs_storage) [
            datastore, 
            right of=source_data, 
            xshift=2cm] {Хранилище протокола работы};
        \node (sample_extract) [
            process, 
            right of=source_logs_storage, 
            xshift=1.5cm, 
            yshift=2cm] {Проведение выборки};
        \node (reports_extract) [
            process, 
            right of=source_logs_storage, 
            xshift=1.5cm, 
            yshift=-2cm] {Получение статистики метрик};
        \node (sample_report_storage) [
            datastore, 
            right of=source_logs_storage, 
            xshift=4cm] {Хранилище выборки и статистики метрик};
        \node (naive_forecast) [
            process, 
            right of=sample_report_storage, 
            xshift=1.75cm, 
            yshift=3cm] {Построение прогноза протокола работы};
        \node (forecast) [
            process, 
            right of=sample_report_storage, 
            xshift=1.75cm, 
            yshift=-3cm] {Построение прогноза метрик};
        \node (naive_storage) [
            datastore, 
            right of=sample_report_storage, 
            xshift=4cm] {Хранилище прогноза протокола работы и метрик};
        \node (calibration) [
            process, 
            below of=naive_storage, 
            yshift=-4.5cm] {Калибровка};
        \node (forecast_storage) [
            datastore, 
            left of=calibration, 
            xshift=-5cm] {Хранилище результатов калибровки};
        \node (result) [
            source, 
            left of=forecast_storage, 
            xshift=-5cm] {Итоговый прогноз протокола работы};
        \draw [arrow] (source_data) -- (source_logs_storage);
        \draw [arrow] (source_logs_storage)--($(source_logs_storage.east)-(0,0.5)$) -| (reports_extract);
        \draw [arrow] (source_logs_storage)--($(source_logs_storage.east)-(0,-0.5)$) -| (sample_extract);
        \draw [arrow] (reports_extract) -| (sample_report_storage);
        \draw [arrow] (sample_extract) -| (sample_report_storage); 
        \draw [arrow] (sample_report_storage)--($(sample_report_storage.east)-(0,-0.5)$) -| (naive_forecast);
        \draw [arrow] (sample_report_storage)--($(sample_report_storage.east)-(0,0.5)$) -| (forecast);
        \draw [arrow] (naive_forecast) -| (naive_storage);
        \draw [arrow] (forecast) -| (naive_storage);
        \draw [arrow] ($(naive_storage.east)-(0,0)$) -- ++(0.5,0) node(lowerright){} |- (calibration.east);
        \draw [arrow] (calibration) -- (forecast_storage);
        \draw [arrow] (forecast_storage) -- (result);
    \end{tikzpicture}
    }
\end{figure}
\end{frame}

\begin{frame}
    \frametitle{Используемые технологии разработки}
    \begin{block}{Основной язык программирования}
        Python 3
    \end{block}
    \begin{block}{Библиотеки математического моделирования}
        Facebook Prophet (для прогнозирования метрик), Scipy, Pandas, Numpy (для обработки данных и решения матрицы калибровки)
    \end{block}
    \begin{block}{Системы обработки больших данных}
        Apache Hadoop, Apache Spark, Apache Mesos
    \end{block} 
\end{frame}

\begin{frame}
    \frametitle{Описание исходных данных}
    \begin{block}{Общее число взаимодействий}
        $\approx 10^{9}$
    \end{block}
    \begin{block}{Хронологический период}
        1 марта 2018 года -- 1 мая 2019 года
    \end{block}
    \begin{block}{Среднее число взаимодействий в день}
        $\approx 10^{5}$
    \end{block}
    \begin{block}{Среднее число уникальных пользователей в день}
        $\approx 10^{4}$
    \end{block}
    \begin{block}{Число уникальных рекламных элементов}
        $1525$
    \end{block}
\end{frame}

\begin{frame}
    \frametitle{Описание исходных данных}
    \begin{figure}[ht]
        \centering
        \begin{tikzpicture}[font=\small]
            \begin{axis}[
                /pgf/number format/.cd,
                    fixed,
                    use comma,
                /tikz/.cd,
                extra x ticks={1/7,1/3.5},
                extra x tick style={
                    grid=major,
                    tick pos = right,
                    ticklabel pos = right,
                },
                xmin=0,
                xmax=0.5,
                ymin=-0.1,
                ylabel={Спектральная плотность $S$},
                xlabel={Частота},
                width=.8\textwidth,
                tick scale binop=\times,
                axis x line*=bottom,
                axis y line*=left,
                xlabel near ticks,
                ylabel near ticks,]
              \addplot+[ycomb, thick, mark=*] table [x=freq, y=value] {\fourierdataset};
            \end{axis}
        \end{tikzpicture}
    \end{figure}
\end{frame}

\end{document}