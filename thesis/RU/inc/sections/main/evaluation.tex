\section{Численные экперименты прогнозирования}
\subsection{Описание исходных данных}
В качестве исходных данных используется протокол работы рекламной системы, который содержит данные о взаимодействиях
пользователей одной рекламной сети. Исходные данные обфусцированы и не содержат информации, представляющей
персональные данные пользователей интернет-страницы.

Общее описание исходных данных представлено в таблице \ref{tab:source-data-review}.
\setlength\LTleft{0pt}
\setlength\LTright{0pt}
\begin{longtable}{@{\extracolsep{\fill}}|x{0.45\textwidth}|x{0.45\textwidth}|}
        \caption{Общее описание исходных данных} \label{tab:source-data-review} \tn
        \hline
        Параметр исходных данных & Значение\tn\hline
        \endfirsthead
        \caption*{Продолжение таблицы~\thetable}\tn\hline 
        Параметр исходных данных & Значение\tn\hline
        \endhead
        Общее число взаимодействий & $1,28 \times 10^{11}$ \tn\hline
        Среднее число взаимодействий в день & $ 2,3 \times 10^5 $ \tn\hline
        Среднее число уникальных пользователей в день & $ 8,5 \times 10^4 $ \tn\hline
        Число уникальных элементов & 1525 \tn\hline
\end{longtable}

График метрики общего числа взаимодействий пользователей~\eqref{eq:net-impressions-def} для периода снятия результатов в 1 день,
построенный на обработанных данных предоставленного протокола работы рекламной системы представлен на рисунке~\ref{img:network-impressions}.
\begin{figure}[ht!]
    \centering
    \begin{tikzpicture}[font=\small]
        \begin{axis}[
            /pgf/number format/.cd,
                fixed,
                use comma,
            /tikz/.cd,
            date coordinates in=x,
            width=.95\textwidth,
            height=.32\textwidth,
            xmin=2018-03-01,
            xmax=2019-03-01,
            xtick distance=90,
            tick scale binop=\times,
            xlabel=Дата,
            ylabel style={align=center},
            ylabel=Число\\взаимодействий]
            \addplot+[no markers] table[x=date, y=impressions] {\networkimpressions};
        \end{axis}
    \end{tikzpicture}
    \caption{Метрика общего числа взаимодействий пользователей с рекламными элементами}\label{img:network-impressions}
\end{figure}


Для вероятностной выборки значение вероятности $\pi$ выбрано исходя из вычислительной мощности ЭВМ и предполагаемой оценке времени
построения прогноза (3--4 часа) и равно 0,001. Верхняя и нижняя границы решения калибровочной системы линейный уравнений выбраны 500 и 1500
согласно требованиям предметной области и снижения влияния выбросов в исходных данных.

Для построения графика спектральной плотности используется временной ряд разностей первого порядка, представленный выражением \eqref{eq:firstdiffs}.
Данное действие необходимо для удаления трендовой составляющей из исходной статистики процесса.
\begin{equation}
    \Delta y (t) = y\left(t\right) - y\left(t-1\right).
    \label{eq:firstdiffs}
\end{equation}

Спектральный анализ исходных данных, использованный в при построении прогноза изображен на рисунке~\ref{img:spectraldensity}.
Очевидны гармоники 7 дневной сезонности (частота $\approx 0.14\;\text{дня}^{-1}$). 
\begin{figure}[ht]
    \centering
    \begin{tikzpicture}[font=\small]
        \begin{axis}[
            /pgf/number format/.cd,
                fixed,
                use comma,
            /tikz/.cd,
            extra x ticks={1/7,1/3.5},
            extra x tick style={
                grid=major,
                tick pos = right,
                ticklabel pos = right,
            },
            xmin=0,
            xmax=0.5,
            ymin=-0.1,
            ylabel={Спектральная плотность $S$},
            xlabel={Частота},
            width=.8\textwidth,
            tick scale binop=\times,
            axis x line*=bottom,
            axis y line*=left,
            xlabel near ticks,
            ylabel near ticks,]
          \addplot+[ycomb, thick, mark=*] table [x=freq, y=value] {\fourierdataset};
        \end{axis}
    \end{tikzpicture}
    \caption{Спектральная плотность исходных данных}\label{img:spectraldensity}
\end{figure}

\subsection{Методика оценки качества алгоритма прогнозирования}
Так как целью разработки алгоритма прогнозирования взаимодействий пользователей с рекламными элементами интернет-страниц
было повышение качества конечного прогноза, необходимо оценить результат работы алгоритма прогнозирования
на различных наборах данных.

Было произведено 3 численных эксперимента, при проведении которых производилась вариация параметров прогноза, таких как
дата прогноза, число исходных данных и длительность прогноза.

Для определения качества алгоритма прогнозирования взаимодействий пользователей с рекламными элементами интернет-страниц
производится сравнение метрики количества взаимодействий пользователей с отдельно взятыми рекламными 
элементами \eqref{eq:adunit-impressions} для собранной статистики (реальных значений) с оценкой прогноза данной метрики,
полученной при помощи разработанной модели прогнозирования.

Для измерения точности прогноза используется следующий подход~\autocite{eval:metrics}. Пусть известно реальное значение процесса (в нашем случае
реальное значение метрики \eqref{eq:adunit-impressions} или метрики \eqref{eq:net-impressions-def}) $y_1, \dots, y_T$.
Разделим исходные сведения о процессе на набор данных для обучения $y_1, \dots, y_N$ и набор данных для оценки прогноза 
$y_{N+1}, \dots, y_T$. Пусть $\hat{y}_{t+h}$ оценка прогноза процесса для периода прогнозирования $h$, а $y_{t+h}$ -- реальное
значение процесса в момент времени $t+h$, тогда можно ввести абсолютную ошибку прогноза, как 
\begin{equation}
    e_t = y_t - \hat{y}_t,\; t = \overline{N+1, T}.
\end{equation}

Для определения средней абсолютной ошибки прогноза метрики~\eqref{eq:net-impressions-def} воспользуемся выражением для
средней абсолютной ошибки
\begin{equation}
    \text{MAE} = \text{mean}\left( \left| e_t \right| \right),\; t = \overline{N+1, T}.
\end{equation}
и среднеквадратичной ошибки
\begin{equation}
    \text{RMSE} = \sqrt{\text{mean}\left( e_t^2 \right)},\; t = \overline{N+1, T}.
\end{equation}

Для оценки качества прогноза метрики~\eqref{eq:adunit-impressions} и~\eqref{eq:net-impressions-def} воспользуемся определением функции
ошибки
\begin{equation}
    p_t = \dfrac{e_t}{y_t}.
\end{equation}

Тогда выражение для относительной средней ошибки будет выглядеть следующим образом
\begin{equation}
    \text{MAPE} = \text{mean}\left( \left| p_t \right| \right),\; t = \overline{N+1, T}.
\end{equation}

В качестве визуальной оценки качества прогноза используется диаграмма рассеяния~\autocite{eval:scatter}.

\subsection{Численные эксперименты построения прогноза взаимодействий пользователей с рекламными элементами интернет-страниц}

В качестве первого эксперимента было рассмотрено прогнозирование с использованием исходных данных с 1 марта 28 года по 28 февраля 2019 года
на 1, 7 и 14 дней. Диаграмма рассеяния изображена на рисунке \ref{img:scatter1-1}.
\begin{figure}[ht!]
    \centering
    \begin{tikzpicture}[font=\small]
        \begin{axis}[
            unit vector ratio*=1 1 1,
            width=0.6\linewidth,
            xlabel=Предсказанное значение,
            ylabel=Наблюдаемое значение,
            legend style={
                at={(0.5,1.1)},
                anchor=south,
                nodes={right},
                font=\mystrut},
            legend columns=2,
            xmax=1e7,
            ymin=-0.01,
            xmin=-0.01,
            tick scale binop=\times,
            ymax=0.9e7,
            xmax=0.9e7]
        \addplot+[
            only marks,
            mark=*,
            opacity=0.3,
            ]
        table[x=real, y=naive] {\fbteod};
        \addplot+[
            only marks,
            mark=*,
            opacity=0.3]
        table[x=real, y=calibrated] {\fbteod};
        \addplot[
            transparent,
            no markers,
            name path=A,
            domain=0:1e8
        ]{1.1*x};
        \addplot[
            transparent,
            no markers,
            name path=B,
            domain=0:1e8
        ]{0.9*x};
        \addplot [
            blue,
            area legend,
            opacity=0.2,
            ] fill between [
                of=A and B
            ];
        \addplot [
            transparent,
            no markers,
            name path=C,
            domain=0:1e8
        ]{1.2   *x};
        \addplot [
            transparent,
            no markers,
            name path=D,
            domain=0:1e8
        ]{0.8*x};
        \addplot [
            blue,
            area legend,
            opacity=0.1,
            ] fill between [
                of=C and D
            ];
        \legend{
            Предложенный алгоритм,
            Наивный подход,
            ,
            ,
            Область ошибки $10\%$,
            ,
            ,
            Область ошибки $20\%$,
        };
        \end{axis}
    \end{tikzpicture}
    \caption{График рассеяния прогнозов (горизонт прогноза 1 день)}\label{img:scatter1-1}
\end{figure}
\begin{figure}[ht!]
    \centering
\begin{tikzpicture}[font=\small]
    \begin{axis}[
        ybar,
        width=0.6\linewidth,
        ylabel=Число отсчетов прогноза,
        legend style={at={(0.5,1.1)},anchor=south},
        legend columns=2,
        tick scale binop=\times]
        \addplot+[
            hist={
                bins=11,
                data min=-2e6,
                data max=2e6
            },
            opacity=0.5   
        ] table [y index=0] {\fbteodmaecalibrated};
        \addplot+[
            hist={
                bins=11,
                data min=-2e6,
                data max=2e6
            },
            opacity=0.5   
        ] table [y index=0] {\fbteodmaenaive};
    \legend{Предложенный алгоритм, Наивный подход}
    \end{axis}
\end{tikzpicture}
\caption{Распределение абсолютной ошибки (горизонт прогноза 1 день)}\label{img:1-1-mae}
\end{figure}
\begin{figure}[ht!]
    \centering
    \begin{tikzpicture}[font=\small]
    \begin{axis}[
        ybar,
        width=0.6\linewidth,
        ylabel=Число отсчетов прогноза,
        legend style={at={(0.5,1.1)},anchor=south},
        legend columns=2,
        tick scale binop=\times]
        \addplot+[
            hist={
                bins=11,
                data min=-1,
                data max=1
            },
            opacity=0.5   
        ] table [y index=0] {\fbteodmapecalibrated};
        \addplot+[
            hist={
                bins=11,
                data min=-1,
                data max=1
            },
            opacity=0.5   
        ] table [y index=0] {\fbteodmapenaive};
    \legend{Предложенный алгоритм, Наивный подход}
    \end{axis}
\end{tikzpicture}
\caption{Распределение относительной ошибки (горизонт прогноза 1 день)}\label{img:1-1-mape}
\end{figure}
\begin{figure}[ht]
    \centering
    \begin{tikzpicture}[font=\footnotesize]
        \begin{axis}[
            unit vector ratio*=1 1 1,
            width=0.8\linewidth,
            xlabel=Предсказанное значение,
            ylabel=Наблюдаемое значение,
            legend style={
                at={(0.5,1.1)},
                anchor=south,
                nodes={right},
                font=\mystrut},
            legend columns=2,
            xmax=1e7,
            ymin=-0.01,
            xmin=-0.01,
            tick scale binop=\times,
            ymax=0.9e7,
            xmax=0.9e7]
        \addplot+[
            only marks,
            mark=*,
            opacity=0.3,
            ]
        table[x=real, y=naive] {\fbtesd};
        \addplot+[
            only marks,
            mark=*,
            opacity=0.3]
        table[x=real, y=calibrated] {\fbtesd};
        \addplot[
            transparent,
            no markers,
            name path=A,
            domain=0:1e8
        ]{1.1*x};
        \addplot[
            transparent,
            no markers,
            name path=B,
            domain=0:1e8
        ]{0.9*x};
        \addplot [
            blue,
            area legend,
            opacity=0.2,
            ] fill between [
                of=A and B
            ];
        \addplot [
            transparent,
            no markers,
            name path=C,
            domain=0:1e8
        ]{1.2   *x};
        \addplot [
            transparent,
            no markers,
            name path=D,
            domain=0:1e8
        ]{0.8*x};
        \addplot [
            blue,
            area legend,
            opacity=0.1,
            ] fill between [
                of=C and D
            ];
        \legend{
            Предложенный алгоритм,
            Наивный подход,
            ,
            ,
            Область ошибки $10\%$,
            ,
            ,
            Область ошибки $20\%$,
        };
        \end{axis}
    \end{tikzpicture}
    \caption{График рассеяния прогнозов метрики взаимодействия пользователей с рекламными элементами интернет-страниц (горизонт прогноза 3 дня)}\label{img:scatter1-3}
\end{figure}
\begin{figure}[ht]
    \centering
\begin{tikzpicture}[font=\footnotesize]
    \begin{axis}[
        ybar,
        width=0.6\linewidth,
        ylabel=Абсолютная ошибка,
        legend style={at={(0.5,1.1)},anchor=south},
        legend columns=2,
        tick scale binop=\times]
        \addplot+[
            hist={
                bins=11,
                data min=-2e6,
                data max=2e6
            },
            opacity=0.5   
        ] table [y index=0] {\fbtefdmaecalibrated};
        \addplot+[
            hist={
                bins=11,
                data min=-2e6,
                data max=2e6
            },
            opacity=0.5   
        ] table [y index=0] {\fbtefdmaenaive};
    \legend{Предложенный алгоритм, Наивный подход}
    \end{axis}
\end{tikzpicture}
\caption{Распределение абсолютной ошибки (горизонт прогноза 3 дня)}\label{img:1-3-mae}
\end{figure}
\begin{figure}[ht!]
    \centering
\begin{tikzpicture}[font=\small]
    \begin{axis}[
        ybar,
        width=0.6\linewidth,
        ylabel=Число отсчетов прогноза,
        legend style={at={(0.5,1.1)},anchor=south},
        legend columns=2,
        tick scale binop=\times]
        \addplot+[
            hist={
                bins=11,
                data min=-1,
                data max=1
            },
            opacity=0.5   
        ] table [y index=0] {\fbtefdmapecalibrated};
        \addplot+[
            hist={
                bins=11,
                data min=-1,
                data max=1
            },
            opacity=0.5   
        ] table [y index=0] {\fbtefdmapenaive};
    \legend{Предложенный алгоритм, Наивный подход}
    \end{axis}
\end{tikzpicture}
\caption{Распределение относительной ошибки (горизонт прогноза 3 дня)}\label{img:1-3-mape}
\end{figure}
\begin{figure}[ht!]
    \centering
    \begin{tikzpicture}[font=\small]
        \begin{axis}[
            unit vector ratio*=1 1 1,
            width=0.65\linewidth,
            xlabel=Предсказанное значение,
            ylabel=Наблюдаемое значение,
            legend style={
                at={(0.5,1.1)},
                anchor=south,
                nodes={right},
                font=\mystrut},
            legend columns=2,
            xmax=1e7,
            ymin=-0.01,
            xmin=-0.01,
            tick scale binop=\times,
            ymax=0.9e7,
            xmax=0.9e7]
        \addplot+[
            only marks,
            mark=*,
            opacity=0.3,
            ]
        table[x=real, y=naive] {\fbtesd};
        \addplot+[
            only marks,
            mark=*,
            opacity=0.3]
        table[x=real, y=calibrated] {\fbtesd};
        \addplot[
            transparent,
            no markers,
            name path=A,
            domain=0:1e8
        ]{1.1*x};
        \addplot[
            transparent,
            no markers,
            name path=B,
            domain=0:1e8
        ]{0.9*x};
        \addplot [
            blue,
            area legend,
            opacity=0.2,
            ] fill between [
                of=A and B
            ];
        \addplot [
            transparent,
            no markers,
            name path=C,
            domain=0:1e8
        ]{1.2   *x};
        \addplot [
            transparent,
            no markers,
            name path=D,
            domain=0:1e8
        ]{0.8*x};
        \addplot [
            blue,
            area legend,
            opacity=0.1,
            ] fill between [
                of=C and D
            ];
        \legend{
            Предложенный алгоритм,
            Наивный подход,
            ,
            ,
            Область ошибки $10\%$,
            ,
            ,
            Область ошибки $20\%$,
        };
        \end{axis}
    \end{tikzpicture}
    \caption{График рассеяния прогнозов (горизонт прогноза 7 дней)}\label{img:scatter1-7}
\end{figure}
\begin{figure}[ht]
    \centering
\begin{tikzpicture}[font=\footnotesize]
    \begin{axis}[
        ybar,
        width=0.6\linewidth,
        ylabel=Абсолютная ошибка,
        legend style={at={(0.5,1.1)},anchor=south},
        legend columns=2,
        tick scale binop=\times]
        \addplot+[
            hist={
                bins=11,
                data min=-2e6,
                data max=2e6
            },
            opacity=0.5   
        ] table [y index=0] {\fbtesdmaecalibrated};
        \addplot+[
            hist={
                bins=11,
                data min=-2e6,
                data max=2e6
            },
            opacity=0.5   
        ] table [y index=0] {\fbtesdmaenaive};
    \legend{Предложенный алгоритм, Наивный подход}
    \end{axis}
\end{tikzpicture}
\caption{Распределение абсолютной ошибки (горизонт прогноза 7 дней)}\label{img:1-7-mae}
\end{figure}
\begin{figure}[ht]
    \centering
\begin{tikzpicture}[font=\footnotesize]
    \begin{axis}[
        ybar,
        width=0.6\linewidth,
        ylabel=Число отсчетов прогноза,
        legend style={at={(0.5,1.1)},anchor=south},
        legend columns=2,
        tick scale binop=\times]
        \addplot+[
            hist={
                bins=11,
                data min=-1,
                data max=1
            },
            opacity=0.5   
        ] table [y index=0] {\fbtesdmapecalibrated};
        \addplot+[
            hist={
                bins=11,
                data min=-1,
                data max=1
            },
            opacity=0.5   
        ] table [y index=0] {\fbtesdmapenaive};
    \legend{Предложенный алгоритм, Наивный подход}
    \end{axis}
\end{tikzpicture}
\caption{Распределение относительной ошибки (горизонт прогноза 7 дней)}\label{img:1-7-mape}
\end{figure}