\section{Алгоритм прогнозирования взаимодействий пользователей с рекламными элементами интернет-страниц}
\subsection{Общая структура алгоритма}
Итоговой целью разработки алгоритма прогнозирования взаимодействий пользователей с рекламными элементами интернет-страниц
является предполагаемы протокол работы рекламной системы, содержащий такую же структуру, как и исходные данные 
(таблица \ref{tab:feature-description}).

Структурно алгоритм представляет собой следующие стадии
\begin{enumerate}
    \item проведение выборки взаимодействий пользователей из начальной совокупности наблюдений;
    \item получение метрик количества взаимодействий и количества уникальных пользователей из начальной совокупности 
    наблюдений;
    \item построение прогноза протокола работы рекламной системы на основе выборкии статистики метрик;
    \item построение прогноза метрик количества взаимодействий и количества уникальных пользователей на основе 
    полученной статистики метрик;
    \item проведение калибровочного процесса для наивного сезонного прогноза протокола работы рекламной системы
    и прогноза метрик количества взаимодействий и количества уникальных пользователей
\end{enumerate}

\subsection{Построение прогноза протокола работы рекламной системы}
Перед построением прогноза протокола работы рекламной системы проводится вероятностная выборка $s$ из начальной совокупности 
наблюдений $\Omega$ с заданной вероятностью $\pi$ для всех событий взаимодействий \eqref{eq:impression-base-with-time}.

Построение прогноза протокола работы рекламной системы производится при помощи модели наивного сезонного 
прогнозирования~\autocite{ml:forecasting}
\begin{equation}
    \hat{\symbf{U}}_{\left.T+h\right|T} = \symbf{U}_{T+h-(k+1)m},
    \label{eq:naive}
\end{equation}
\setlength{\tabcolsep}{0em}\begin{tabular}{@{\hspace*{0em}}m{\parindent}ll}
    где & $\hat{\symbf{U}}_{\left.T+h\right|T}\;$ & {---} прогноз вектора взаимодействия; \\
    & $\symbf{U}$ & {---} известный вектор взаимодействия из выборки; \\
    & $m$ & {---} сезонность исходного процесса; \\
    & $k = \left[\dfrac{h-1}{m}\right]\;$ & {---} целое число. \\
\end{tabular}
\medskip

Для определения параметра $m$ строится амплитудно-частного характеристика метрики взаимодействий исходной статистики
на основе спектрального разложения в ряд Фурье. Значение периода преобладающей гармоники используется в качестве
периода сезонности в модели наивного сезонного прогнозирования.

В процессе построения прогноза важное место занимает признак взаимодействия, отвечающий за уникальный идентификатор
пользователя протокола работы рекламной системы (таблица \ref{tab:feature-description}).

Пользователей, как уже было сказано ранее, можно классифицировать на разовых и постоянных пользователей. Для того, чтобы
учесть данную особенность пользовательского поведения при построении наивного сезонного прогноза происходит процесс
экстраполяции уникальных пользователей -- предсказание пользователей, на основе коэффициента 
удержания~\autocite{online:retention}.

На основе статистики за период равный удвоенному периоду, используемому при построении наивного сезонного прогноза,
определяется принадлежность пользователя к классу разовых или постоянных пользователей в рамках одного рекламного элемента
на основе определения данных классов. Для разовых пользователей происходит замена значения признака уникального идентификатора
на новое значение, для постоянных значение данного признака остается неизменным.

\subsection{Построение прогнозов метрик}
Полученные метрики количества взаимодействий пользователей~\eqref{eq:adunit-impressions} и количества уникальных
пользователей~\eqref{eq:uniques-definition} существуют для каждого рекламного элемента из исходной совокупности и 
представляют собой временные ряды.

Если число рекламных элементов (мощность значений множества признака вектора~\eqref{eq:impression-base-with-time},
отвечающего за идентификатор рекламного элемента) известно и равно $N_A$, то в результате применения 
функции~\eqref{eq:adunit-impressions} к исходной совокупности мы получим набор $N_A$ временных рядов. Аналогичным образом
определяется совокупность временных рядов, представленных метрикой количества уникальных 
пользователей~\eqref{eq:uniques-definition}. Для каждого из $2 \cdot N_A$ временных рядов строится временной прогноз с 
использованием модели~\eqref{eq:prophet}.

\subsection{Проведение калибровочного процесса}
Наивный сезонный прогноз протокола работы рекламной системы и построенные прогнозы метрик числа взаимодействий и числа
уникальных пользователей позволяют осуществить процесс калибровки. Основная идея данного шага алгоритма прогноза заключается
в получении оценок метрик числа взаимодействий и количества уникальных пользователей, полученных путем применения 
функций~\eqref{eq:adunit-impressions} и~\eqref{eq:uniques-definition} к прогнозу~\eqref{eq:naive} через
вспомогательную информацию о прогнозе тех же метрик, полученных путем применения модели~\eqref{eq:prophet}.

Наивный сезонный прогноз протокола работы рекламной системы содержит данные за отрезок времени $\left[T_0, T_1\right]$.
Построим разбиение данного временного отрезка на отдельные временные отрезки продолжительностью $\Delta T$, тогда исходный
отрезок времени $\left[T_0, T_1\right]$ можно записать в виде
\begin{equation}
    \Delta = \left[T'_0, T'_1\right) \cup \left[T'_1, T'_2\right)
    \cup \dots \cup \left[T'_{K-2}, T'_{K-1}\right) \cup \left[T_{K-1}, T_K\right], T'_{k+1} - T'_{k} = \Delta T
\end{equation}

Для каждого временного разбиения из $\left[T'_k, T'_{k+1}\right)$ каждого пользователя $U_i$ и рекламного элемента 
$A_j$ построим метрику~\eqref{eq:user-period-imps-definition}. Для этого сначала применим 
функцию~\eqref{eq:user-imps-definition} к наивному сезонному прогноза протокола взаимодействий, а к полученному набору 
значений применим функцию \eqref{eq:user-period-imps-definition} для каждого из временных разбиений. Аналогичную 
операцию проведем и для метрики \eqref{eq:uniques-definition} количества уникальных взаимодействий.

Если в качестве индекса строки взять пару значений временного отрезка и идентификатора рекламного элемента 
$\left( \left[ T'_k, T'_{k+1} \right), A_j \right)$, а идентификатор пользователя $U_i$ взять в качестве то 
набор метрик, представленных выражением \eqref{eq:user-period-imps-definition}, является матрицей взаимодействий MIMPS
\begin{equation}
    \text{MIMPS} = \text{Imps} \left( U_i, A_j, T_k, T_{k+1}, \left\{ \hat{\symbf{U}} \right\} \right)
    \label{eq:impression-matrix}
\end{equation}
\setlength{\tabcolsep}{0em}\begin{tabular}{@{\hspace*{0em}}m{\parindent}ll}
    где & $U_i$ & {---} идентификатор пользователя; \\
    & $A_j$ & {---} идентификатор рекламного элемента; \\
    & $T_k,\; T_{k+1}\;$ & {---} начало и конец временного отрезка; \\
    & $\hat{\symbf{U}}\;$ & {---} оценка прогноза протокола работы рекламной системы. \\
\end{tabular}
\medskip

Элементы матрицы MIPS отражают количество взаимодействие которое совершил пользователь $U_i$ с рекламным элементом $A_j$
за период времени $\left[T_k, T_{k+1}\right)$.

Если число уникальных пользователей $N_U$ (мощность значений множества признака вектора~\eqref{eq:impression-base-with-time},
отвечающего за идентификатор рекламного элемента) и число уникальных рекламных элементов $N_A$ известно, то матрица MIMPS
будет матрицей положительных целых чисел размер
\begin{equation}
    \text{MIMPS} \in \mathbb{Z}^{+(K \cdot N_A) \times N_U}
\end{equation}
\setlength{\tabcolsep}{0em}\begin{tabular}{@{\hspace*{0em}}m{\parindent}ll}
    где & $K$ & {---} число разбиений временного отрезка; \\
    & $N_A\;$ & {---} число рекламных элементов; \\
    & $N_U\;$ & {---} число уникальных пользователей. \\
\end{tabular}
\medskip

Для получения матрицы уникальных пользователей MDSTN, можно воспользоваться определением матрицы 
MIPS~\eqref{eq:impression-matrix}. Для всех ненулевых элементов матрицы MIPS элементы матрицы MDSTN с тем же индексом
будут содержать значение 1, и 0 для всех остальных.
\begin{equation}
    \text{MDSTN} =
    \begin{cases} 
        1,\; \text{Imps} \left( U_i, A_j, T_k, T_{k+1}, \left\{ \hat{\symbf{U}} \right\} \right) \neq 0 \\
        0,\; \text{Imps} \left( U_i, A_j, T_k, T_{k+1}, \left\{ \hat{\symbf{U}} \right\} \right) = 0
    \end{cases}
    \label{eq:uniques-matrix}
\end{equation}

Матрица~\eqref{eq:uniques-matrix} будет иметь ту же размерность, что и матрица~\eqref{eq:impression-matrix}.
\begin{equation}
    \text{MDSTN} \in \mathbb{Z}^{+(K \cdot N_A) \times N_U}
\end{equation}
\setlength{\tabcolsep}{0em}\begin{tabular}{@{\hspace*{0em}}m{\parindent}ll}
    где & $K$ & {---} число разбиений временного отрезка; \\
    & $N_A\;$ & {---} число рекламных элементов; \\
    & $N_U\;$ & {---} число уникальных пользователей. \\
\end{tabular}
\medskip

Аналогичную операцию можно провести для наборов прогнозов метрик, определенных выражением~\eqref{eq:adunit-impressions}
и~\eqref{eq:uniques-definition}. Если взять в качестве индекса $\left( \left[ T'_k, T'_{k+1} \right), A_j \right)$, то
получим вектора IMPS \eqref{eq:imps-vector}, отражающий число всех пользовательских взаимодействий на рекламном элементе $A_j$ за 
период $\left[ T'_k, T'_{k+1} \right)$
\begin{equation}
    \text{IMPS} = \text{Imps} \left(A_j, T'_k, T'_{k+1}, \left\{ \hat{\symbf{U}} \right\} \right),
    \label{eq:imps-vector}
\end{equation}
и DSTN, отражающий число уникальных пользователей, взаимодействующих с рекламным элементом $A_j$ за период 
$\left[ T'_k, T'_{k+1} \right)$
\begin{equation}
    \text{DSTN} = \text{Distinct} \left(A_j, T'_k, T'_{k+1}, \left\{ \hat{\symbf{U}} \right\} \right).
    \label{eq:uniques-vector}
\end{equation}

IMPS и DSTN являются векторами положительных целых чисел размерности $\mathbb{Z}^{+K\cdot N_A}$.

Задача калибровки \eqref{eq:strat-greg} после всех преобразований заключается в поиске решения системы линейных
алгебраических уравнений \eqref{eq:system}
\begin{equation}
    \left(
        \begin{array}{ccc}
            \text{MIMPS}_{00} & \dots & \text{MIMPS}_{0N_U} \\
            \vdots & \ddots & \vdots \\
            \text{MIMPS}_{(K\cdot N_A)0} & \dots & \text{MIMPS}_{(K\cdot N_A)N_U} \\
            \text{MDSTN}_{00} & \dots & \text{MDSTN}_{0N_U} \\
            \vdots & \ddots & \vdots \\
            \text{MDSTN}_{(K\cdot N_A)0} & \dots & \text{MDSTN}_{(K\cdot N_A)N_U} \\
        \end{array}
    \right)\times
    \left(
        \begin{array}{c}
            \text{us}_{0} \\
            \vdots \\
            \vdots \\
            \vdots \\
            \vdots \\
            \text{us}_{N_U}
        \end{array}
    \right) = 
    \left(
        \begin{array}{c}
            \text{IMPS}_{00} \\
            \vdots \\
            \text{IMPS}_{K\cdot N_A} \\
            \text{DSNT}_{0} \\
            \vdots \\
            \text{DSNT}_{(K\cdot N_A)0}
        \end{array}
    \right)
    \label{eq:system}
\end{equation}

Вектор решения $\text{us}$ системы линейных уравнений~\eqref{eq:system} является решением задачи 
калибровки~\eqref{eq:calib-equation}.

На решение системы уравнений~\eqref{eq:system} накладываются требования, которые обусловлены пользовательским
поведением
\begin{enumerate}
    \item компоненты вектора решения us являются строго положительными (строго говоря, значение компоненты 
    $i$ вектора $us$ отражает число пользователей с параметрами соответствующими пользователю $i$ из
    прогноза протокола работы рекламной системы);
    \item компоненты вектора решения us ограничены снизу и сверху положительными константами $\text{us}_l$ и 
    $\text{us}_u$, которые синтетически ограничивают минимальное и максимальное значение прогнозируемых
    метрик числа взаимодействий и числа уникальных пользователей (такое ограничение позволяет минимизировать
    влияние выбросов в исходных данных, которые могут попасть в прогноз при построении наивного прогноза
    протокола работы рекламной системы).
\end{enumerate}

Чтобы удовлетворить данные требования для решения системы \eqref{eq:system} используется алгоритм TRF 
(англ. Trust Region Reflective)~\autocite{algo:trf}.