\begin{center}
МИНИСТЕРСТВО НАУКИ И ВЫСШЕГО ОБРАЗОВАНИЯ \\
РОССИЙСКОЙ ФЕДЕРАЦИИ\vspace{14pt} \\
федеральное государственное автономное \\
образовательное учреждение высшего образования \\
<<Самарский национальный исследовательский университет \\
имени академика С.П. Королева>> \\
(Самарский университет)\vspace{14pt} \\
Институт информатики, математики и электроники \\
Факультет информатики \\
Кафедра информационных систем и технологий\vspace{28pt} \\
\end{center}
\begin{flushright}
    \begin{tabular}{@{}c@{}}
        \onehalfspacing{}
        УТВЕРЖДАЮ\\
        Заведующий кафедрой ИСТ\\
        \underline{\hspace{4cm}} Прохоров С.А\\
        <<\underline{\hspace{0.75cm}}>> \underline{\hspace{4.13cm}} 20\underline{\hspace{0.75cm}} г.
    \end{tabular}
\end{flushright}\vspace{28pt}
\begin{center}
    ЗАДАНИЕ НА ВЫПУСКНУЮ КВАЛИФИКАЦИОННУЮ РАБОТУ МАГИСТРА
\end{center}\vspace{14pt}
\onehalfspacing{}
студенту Барскову Никите Максимовичу \\
группа \textnumero 6223-090401D. \\
Тема работы: прогнозирование взаимодействий пользователей с рекламными элементами интернет-страниц \\
утверждена приказом по университету \textnumero 81 от 23 января 2019 г. \\
Исходные данные к работе: \\
протокол работы системы покупки-продажи интернет-рекламы, содержащий обезличенные данные о взаимодействии пользователей с
рекламными элементами интернет-страниц.\\
Перечень вопросов, подлежащих разработке в работе: 
\begin{enumerate}
    \item анализ процесса взаимодействия пользователей с рекламными элементами интернет-страниц;
    \item математическая постановка задачи прогнозирования;
    \item разработка алгоритма прогнозирования взаимодействий пользователей;
    \item программная реализация разработанного алгоритма;
    \item оценка качества работы разработанного алгоритма в сравнении с существующим 
    методом наивного сезонного прогнозирования.
\end{enumerate}
\singlespacing{}
Руководитель работы,\\
\tlinefill{к.т.н., доцент кафедры ИСТ}{(подпись)}{И.М. Куликовских}\vspace{14pt}\\
\tlinefill{Задание принял к исполнению}{(подпись)}{Н.М. Барсков}\vspace{14pt}\\
<<\underline{\hspace{0.75cm}}>> \underline{\hspace{4.13cm}} 20\underline{\hspace{0.75cm}} г.