\subsection{Метод калибровки выборки}
\subsubsection{Определение калибровки}

Методика калибровки для конечных совокупностей \autocite{ps:summary} заключается в
\begin{enumerate}
    \item вычислении весов, которые включают определенную вспомогательную информацию и ограничены уравнениями калибровки;
    \item использовании полученных весов, для вычисления параметров конечной совокупности;
    \item цели получить почти не смещенные оценки.
\end{enumerate}

Основная идея получения калиброванных статистических оценок, предложенная~\autocite{ps:calibration}, представляет собой
расчет калиброванных весов выборки при условия ограничения вида: сумма весов первоначального плана выборки и сумма
калиброванных весов -- равны. Калибровка обеспечивает простой практический подход к включению вспомогательной информации
в оценку~\autocite{ps:estimation}.

\subsubsection{Обобщенная регрессия. Метод минимального расстояния}

Пусть дана конечная совокупность $\Omega = \left\{1, 2, \dots, i, \dots, N \right\}$ из $N$ наблюдений из которой выполнена
вероятностная выборка $s, s\in\Omega$ размера $n$, получена с вероятностью $p(s)$ в соответствии с планом выборки $p$.
Вероятности включения объекта в выборку  $\pi_i = \text{Pr}\left( i \in s\right)$ известны.

Пусть $y_i$ есть значений интересующей переменной $y$ для $i$-го наблюдения совокупности, с которым связана 
вспомомогательная переменная $x_i$. Для объекта из выборки $s$ значения $\left(y_i, x_i\right)$ известны. Цель составить
оценку неизвестной суммы по совокупности $\sum\limits_{i \in \Omega} y_i$. 

Для производимой калибровки важно точно определить вспомогательную информацию. Метод оценивания с помощью обобщенной
регрессии (англ. GREG, generalized regression estimation) есть систематизированный способ принять во внимание
вспомогательную информацию.

Простая линейная оценка GREG-оценка описывается и в~\autocite{ps:surveysampling} и ее центральная идея заключается
в том, что предсказанные $\hat{y}_i$ могут быть вычислены для всех $N$ совокупности, с помощью подобранной вспомогательной
модели и использовании вспомогательных значений $x_i$.

Сумма $Y = \sum\limits_{i \in \Omega} y_i$ определяется в \eqref{eq:strat-greg}:
\begin{equation}
    \hat{Y}_{\text{GREG}} = \sum\limits_{\Omega}\hat{y}_i + \sum\limits_{s}d_i\left(y_i - \hat{y}_i\right)
    \label{eq:strat-greg}
\end{equation}
\setlength{\tabcolsep}{0em}\begin{tabular}{@{\hspace*{0em}}m{\parindent}ll}
    где & $y_i\;$ & {---} интересующая переменная; \\
    & $\hat{y}_i\;$ & {---} оценка интересующей переменной; \\
    & $d_i$ & {---} калибровочный вес. \\
\end{tabular}
\medskip

Цель создания такой конструкции это построение оценки $\hat{Y}_{\text{GREG}}$ с помощью подбора вспомогательной
модели, которая дает малую невязку $y_i - \hat{y_i}$.

Можно записать линейную оценку GREG \eqref{eq:strat-greg} как взвешенную выборочную сумму 
$\hat{Y}_{\text{GREG}} = \sum\limits_{s}w_iy_i$ с весами
\begin{equation}
    w_i = d_ig_i,
    \; g_i = 1 + q_i\symbf{\lambda}\symbf{x}_i,
    \; \symbf{\lambda} = \left( \sum\limits_{\Omega} x_i - \sum\limits_{s}d_ix_i \right)
    \left(\sum\limits_sd_iq_i\symbf{x}_i\symbf{x}_i'\right)^{-1}
    \label{eq:calib-equation}
\end{equation}
\setlength{\tabcolsep}{0em}\begin{tabular}{@{\hspace*{0em}}m{\parindent}ll}
    где & $w_i\;$ & {---} калибровочные веса; \\
    & $q_i\;$ & {---} масштабные коэффициенты; \\
    & $x_i$ & {---} значения вспомогательной переменной. \\
\end{tabular}
\medskip

Для подбора значений весов $w_i$ используется метод минимального расстояния, который предполагает изменять 
начальные веса $d_i$ к новым весам $w_i$, определенным как близкие к $d_i$. Для определения степени близости 
используется функция расстояния $G\left(w, d\right)$.

Минимизируя сумму расстояний $\sum\limits_sG_i\left(w_i, d_i\right)$ на ограничениях уравнений калибровки 
$\sum\limits_{s} w_i \symbf{x}_i = \sum\limits_{\Omega} \symbf{x}_i$ мы можем получить значения калибровочных весов.