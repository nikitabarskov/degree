\subsection{Алгоритмическая покупка рекламы}
\subsubsection{Основные термины и понятия}

Алгоритмическая покупка рекламы --- это способ автоматизированной доставки цифрового рекламного содержимого,
например, рекламного содержимого интернет-страниц, в режиме реального времени, основанный на индивидуальных 
особенностях каждого отдельного события доставки. (Oliver Busch Editor Programmatic Advertising)

\tikzstyle{entity} = [
    rectangle,  
    minimum width=3cm, 
    minimum height=1cm,
    font=\footnotesize,
    text centered, 
    text width=2.5cm,
    draw=black,
    fill=blue!15]

\tikzstyle{arrow} = [
    thick,
    ->,
    >=stealth]

\begin{figure}[h!]
    \centering
    \begin{tikzpicture}
        \node (user_1) [entity] {Посетитель 1};
        \node (user_2) [entity, below of=user_1, yshift=-0.5cm] {Посетитель 2};
        \node (user_3) [entity, below of=user_2, yshift=-0.5cm] {Посетитель 3};
        \node (page) [entity, right of=user_2, xshift=2.5cm] {Интернет страница с рекламным элементом};
        \draw [arrow] (user_1) -| node[anchor=south, font=\footnotesize, text width=3cm] {Уведомление о показе рекламного места} (page);
        \draw [arrow] (user_2) -- (page);
        \draw [arrow] (user_3) -| (page);

        \node (ssp) [entity, right of=page, xshift=2.5cm] {SSP};
            
        \draw [arrow] (page) -- (ssp);
            
        \node (dsp_2) [entity, right of=ssp, xshift=2.5cm] {DSP 2};
        \node (dsp_1) [entity, above of=dsp_2, yshift=0.5cm] {DSP 1};
        \node (dsp_3) [entity, below of=dsp_2, yshift=-0.5cm] {DSP 3};

        \draw [arrow] (ssp) |- node[anchor=south, font=\footnotesize, text width=3cm] {Предоставление рекламного места для продажи} (dsp_1);
        \draw [arrow] (ssp) -- (dsp_2);
        \draw [arrow] (ssp) |- (dsp_3);

        \node (advertiser_1) [entity, right of=dsp_1, xshift=2.5cm, yshift=1cm] {Рекламодатель 1};
        \node (advertiser_2) [entity, right of=dsp_2, xshift=2.5cm] {Рекламодатель 2};
        \node (advertiser_3) [entity, right of=dsp_3, xshift=2.5cm, yshift=-1cm] {Рекламодатель 3};

        \draw [arrow] (dsp_1) |- node[anchor=south, font=\footnotesize, text width=3cm] {Запрос рекламного содержимого} (advertiser_1);
        \draw [arrow] (dsp_2) -- (advertiser_2);
        \draw [arrow] (dsp_3) |- (advertiser_3);
    \end{tikzpicture}
    \caption{Схема алгоритмической покупки рекламы}
    \label{img:pragrammatic-advertising}
\end{figure}


На рисунке~\ref{img:pragrammatic-advertising} отображена схема одной из моделей реализации 
алгоритмической покупки рекламы RTB (Real-Time Bidding, покупка в реальном времени). При посещении пользователем интернет ресурса 
с рекламным элементом, браузер собирает доступную о пользователе информацию 
(cookie информацию, техническую информацию об устройстве, местоположении) и отправляет запрос в систему SSP 
(уведомляет SSP о показе рекламного элемента интернет-страницы пользователю).

SSP (Supply Side Platform) --- комплекс аппаратных и программных средств, предназначенный для продажи 
рекламного места.

Рекламный элемент (Advertising Unit, AdUnit) --- набор элементов интернет страницы, предназначенный для 
отображения рекламного содержимого и предоставления возможности взаимодействия с ним.

SSP на основании переданной браузером информации объявляет данный рекламный элемент как доступный для 
продажи на аукционе. Следующим шагом, является уведомление рекламодателей через DSP о возможности покупки
данного рекламного места.

DSP (Demand Side Platform) --- комплекс аппаратных и программных средств, предназначенных для предоставления
рекламного содержимого для дальнейшей продажи.

DSP осуществляют ставки на данное рекламное место, при этом оценивая пользователя на основе предоставленной
SSP информацией и дополняя информацию о пользователе из внешних источников (DMP (Data Managment Platform) --- 
систем управления данными). На конечном этапе рекламное содержимое победителя аукциона доставляется через SSP и
браузер конечному пользователю путем показа на рекламного месте (рекламном элементе).

Взаимодействие пользователя с рекламным элементом (Impression) --- событие просмотра пользователем содержимого
рекламного элемента интернет страницы, сопровождающееся сбором с согласия пользователя сведений, передаваемых 
системам продажи-покупки рекламы.

Число взаимодействий (Impressions amount) --- число взаимодействий пользователей с конкретным рекламным элементом
интернет страницы, осуществленных за определенный период времени.

Число уникальных пользователей (Uniques amount) --- число неповторяющихся пользователей, осуществивших взаимодействие
с конкретным рекламным элементом интернет страницы за определенный период времени.

\subsubsection{Количественная оценка взаимодействий пользователей с рекламными элементами}

Основным источником данных о пользовательских посещениях интернет страниц (информация о пользовательском траффике)
является протокол работы системы продажи-покупки рекламного содержимого. Протокол работы представляет собой файл 
(или совокупность файлов), в котором содержатся данные о взаимодействиях пользователей с рекламным содержимым 
интернет страниц в хронологическом порядке. Чаще всего каждое взаимодействие описывается информацией, собранной 
интернет-обозревателем (веб-браузером), информацией, полученной из внешних источников, и набором технических мета-данных,
сгенерированных в процессе розыгрыша рекламного места. Пример формата данных, содержащихся в данном протоколе, 
описан в таблице \ref{tab:feature-description}.

\tabulinesep = 7pt
\begin{longtabu} to \textwidth {|X|X|X|}
        \caption{Описание признаков взаимодействия}
        \label{tab:feature-description}
        \endfirsthead
        \endhead
        \rowfont[c]{\bfseries}
        \hline
        Название поля & Описание & Множество допустимых значений \\
        \hline
        Идентификатор пользователя
        & Уникальный идентификатор пользователя, присваиваемый ему рекламным сервером
        & Строка, генерируемая согласно внутренней логике рекламного сервера \\
        \hline
        Идентификатор рекламного элемента
        & Уникальный идентификатор рекламного взаимодействия, с которым пользователь взаимодействовал
        & Целое число, генерируемое согласно внутренней логике рекламного сервера \\
        \hline
        Время взаимодействия 
        & Время взаимодействия пользователя с рекламным элементом в формате Unix-time
        & Целое число \\
        \hline
        Идентификатор географического положения пользователя 
        & Уникальный идентификатор географического положения пользователя
        & Целое число, генерируемое согласно внутренней логике рекламного сервера \\
        \hline
        Идентификатор используемого обозревателя интернет страниц 
        & Уникальный идентификатор интернет обозревателя, при помощи которого пользователь совершил
        взаимодействие с рекламным элементом
        & Целое число, генерируемое согласно внутренней логике рекламного сервера \\
        \hline
        Идентификатор операционной системы
        & Уникальный идентификатор операционной системы устройства, при помощи которого пользователь совершил
        взаимодействие с рекламным элементом
        & Целое число, генерируемое согласно внутренней логике рекламного сервера \\
        \hline
\end{longtabu}


Для количественного описания пользовательского поведения служат метрики количества взаимодействий пользователей
с рекламным содержимым интернет страниц и количества уникальных пользователей, взаимодействующий с рекламным
содержимым интернет страниц.

Представим каждое взаимодействие пользователя с рекламным элементов в виде вектора $\mathbf{U}$, элементами которого
являются признаки взаимодействия (данные, содержащиеся в журнале работы рекламной системы).
\begin{equation}
    \mathbf{U_i} = \left(u_0, \dots, u_n \right), i = \overline{1, M}, n = \overline{0, N},
\end{equation}
где $M$ -- общее число взаимодействий, содержащихся в исходном протоколе работы рекламной системы, $N$ -- число признаков
взаимодействия.

В таблице \ref{tab:feature-description} указано, что одним из полей протокола работы рекламной системы является метка
времени взаимодействия. Тогда множество векторов $\mathbf{U_i}$ представляет собой мномерный временной ряд. При этом 
различные события, например, взаимодействие различных пользователей с одним и тем же рекламным элементом могут
происходить в один и тот же момент времени.
\begin{equation}
    \mathbf{U_i} = \left(t_k, u_1, \dots, u_n \right), i = \overline{1, M}, n = \overline{1, N}, k = \overline{1, K},
\end{equation}
где $M$ -- общее число взаимодействий, содержащихся в исходном протоколе работы рекламной системы, $N$ -- число признаков
взаимодействия, $K$ -- число временных отсчетов.

Пусть дана выборка пользователей $\left\{ \mathbf{U} \right\}$ за период времени $\left[T_0, T_1\right]$. Пусть $u_i$
элемент вектора взаимодействия $U$ содержит значение идентификатора пользователя, совершившего взаимодействие, а $u_j$ -- 
индентификатор рекламного элемента, с которым взаимодействие было совершено, тогда
\begin{equation}
    \text{Imps} \left( U, A, t_k, \left\{ \mathbf{U} \right\} \right) =
        \begin{cases}
            1, u_i = U \wedge u_j = A  \\
            0, u_i \neq U \vee u_j \neq A
        \end{cases}, \forall t_k \in \left[T_0, T_1\right]
\end{equation}
функция взаимодействия пользователя $U$ с рекламным элементом $A$ в момент времени $t_k$. 

Метрика количества взаимодействий пользователя $U$ с рекламным элементом $A$ за период $\left[T_0, T_1\right]$ 
будет выглядеть следующим образом
\begin{equation}
    \text{Imps} \left( U, A, T_0, T_1, \left\{ \mathbf{U} \right\} \right) =
    \sum \limits_{t=T_0}^{T_1} \text{Imps} \left( U, A, t, \left\{ \mathbf{U} \right\} \right) 
\end{equation}

Метрика количества взаимодействий является аддитивной величиной. Данное свойство позволяет строить набор производных
метрик. Так, для определения количества пользовательских взаимодействий с рекламным элементом $A$ достаточно просуммировать 
взаимодействия отдельных пользователей
\begin{equation}
    \text{Imps} \left(A, T_0, T_1, \left\{ \mathbf{U} \right\} \right) =
    \sum \limits_{\forall u_i} \text{Imps} \left( u_i, A, T_0, T_1, \left\{ \mathbf{U} \right\} \right)
\end{equation}

Аналогично, для определения всех пользовательских взаимодействий достаточно просуммировать взаимодействия отдельных
рекламных элементов
\begin{equation}
    \text{Imps} \left(T_0, T_1, \left\{ \mathbf{U} \right\} \right) =
    \sum \limits_{\forall u_j} \text{Imps} \left(u_j, T_0, T_1, \left\{ \mathbf{U} \right\} \right)
\end{equation}

Временной ряд, представленный метрикой количества взаимодействий показан на рисунке \eqref{img:impressions}.

\begin{figure}[ht]
    \centering
    \begin{tikzpicture}[font=\footnotesize]
        \begin{axis}[
            /pgf/number format/.cd,
                fixed,
                use comma,
            /tikz/.cd,
            date coordinates in=x,
            width=0.9\textwidth,
            height=0.3\textwidth,
            xmin=2018-03-01,
            xmax=2019-03-01,
            xtick distance=60,
            tick scale binop=\times,
            xlabel=Дата,
            ylabel style={align=center},
            ylabel=Число\\взаимодействий]
            \addplot+[no markers] table[x=date, y=impressions] {\networktimeseries};
        \end{axis}
    \end{tikzpicture}
    \caption{Временной ряд взаимодействий пользователей с рекламным содержимым интернет-страниц}
    \label{img:impressions}
\end{figure}




Другой метрикой, количественно описывающей пользовательское поведение, является метрика количества уникальных пользователей,
взаимодействующих с рекламным элементом.

Пусть дана выборка пользователей $\left\{ \mathbf{U} \right\}$ за период времени $\left[T_0, T_1\right]$. Пусть $u_i$
элемент вектора взаимодействия $U$ содержит значение идентификатора пользователя, совершившего взаимодействие, а $u_j$ -- 
индентификатор рекламного элемента, с которым взаимодействие было совершено, тогда
\begin{equation}
    \text{Distinct}\left( A, T_0, T_1, \left\{\mathbf{U}\right\} \right) =
    \begin{cases}
        \left|\left\{u_i\right\}\right|, u_j = A  \\
        0, u_j \neq A
    \end{cases}, \forall t_k \in \left[T_0, T_1\right]
\end{equation}
метрика количества уникальных пользователей, взаимодействующих с рекламным элементом $A$, где 
$\left|\left\{u_i\right\}\right|$ --- мощность множества уникальных пользователей, которые взаимодействовали с рекламным
элементом $A$ в момент времени $t_k$.

Данная метрика не является аддитивной, так как множество пользователей $\left\{u_i\right\}$ меняется с течением времени.

Временной ряд, представленной метрикой количества уникальных взаимодействий показан на рисунке \eqref{img:uniques}.

\pgfplotstableread[col sep = comma]{inc/csv/network_1_ts.csv}\networktimeseries

\begin{figure}[ht]
    \centering
    \begin{minipage}{0.9\textwidth}    
    \begin{tikzpicture}[font=\scriptsize]
        \begin{axis}[
            date coordinates in=x,
            width=\textwidth,
            height=0.3\textwidth,
            xmin=2018-03-01,
            xmax=2019-03-01,
            xtick distance=60,
            xlabel=Дата,
            ylabel style={align=center},
            ylabel=Число уникальных\\взаимодействий]
            \addplot+[no markers] table[x=date, y=uniques] {\networktimeseries};
        \end{axis}
    \end{tikzpicture}
    \caption{Временной ряд взаимодействий пользователей с рекламным содержимым интернет-страниц}
    \label{img:uniques}
\end{minipage}
\end{figure}



