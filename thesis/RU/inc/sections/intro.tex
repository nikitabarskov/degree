\section{Введение}

Доставка рекламного контента до конечного пользователя является сложным многоэтапным процессом, определяющим
какой конкретный экземпляр рекламы будет показан пользователю в зависимости от его местоположения,
используемых им устройства и программного обеспечения для выхода во Всемирную сеть, истории предыдущих
посещений и прочей передаваемой пользователем информацией (служебной или со сторонних источников). Использование
автоматизированных средств доставки рекламного контента, называется алгоритмической покупкой рекламы (англ.
Programmatic Advertising). Такой подход позволяет минимизировать затраты на человеческий ресурс рекламодателя и
покупателя и уменьшить время доставки контента конечному пользователю.

Прогнозирование взаимодействий пользователей с рекламным содержимым интернет страниц является необходимым этапом
в оценке эффективности и прибыльности рекламных кампаний. На основе предположения о пользовательском поведении
в будущем, параметры рекламной кампании могут оперативно исправлены для достижения большей пользы и лучшего качества.

В данной работе описан и оценен метод решения задачи прогнозирования взаимодействия пользователей с рекламным 
содержимым интернет страниц. Предложенный алгоритм представляет собой комбинацию методов машинного обучения, 
статистики и вычислительной математики, реализованную средствами программирования, библиотек распределенных
вычислений и современных средств обработки и анализа данных. Такой подход позволяет добиться качественной 
автоматизации поставленной задачи и минимизировать ручные затраты инженера для получения результата.

В качестве основного метода прогнозирования используется известный метод наивного сезонного
прогнозирования с автоматическим определением периода сезонности на основе анализа спектральной плотности.
Результаты такого прогноза взвешиваются на основе одномерного прогноза набора метрик путем калибровки. 
Оценка полученных результатов производится на различном наборе исходных данных (бэк-тестирование).

\textbf{Целью работы} является разработка и оценка качества алгоритма прогнозирования взаимодействий пользователей
с рекламным содержимым интернет страниц. Сравнение алгоритма производится с методом наивного сезонного 
прогнозирования.

\textbf{Задачи исследования:}
\begin{enumerate}
    \item Обзор и описание предметной области для применения алгоритма
    \item Постановка задачи прогнозирования
    \item Анализ существующих методов прогнозирования
    \item Математическое описание алгоритма
    \item Реализация алгоритма
    \item Оценка качества разработанного алгоритма
\end{enumerate}