\structuresection{Введение}

История интернет-рекламы берет свое начало 3 мая 1978, когда маркетолог компании DEC (Digital Equipment Corporation) 
Гари Серк (Gary Thuerk) впервые отправил письмо американским пользователям сети ARPANET, содержащее приглашение
на демонстрацию модели нового компьютера DEC-20~\autocite{online:spam}.

За 30 лет развития системы доставки рекламного содержимого в сети Интернет эволюционировали из простых рассылочных
приложений в современные комплексы, реализующие разнообразные способы доставки рекламного содержимого до 
конечного пользователя~\autocite{online:google}.

Требования к системам реализации рекламного содержимого, такие как обработка запросов в реальном времени, 
поддержка широкой аудитории пользователей, персонализация рекламного содержимого способствуют развитию
передовых способов обработки разнородных и неструктурированных данных, появлению новых алгоритмов, разработке
необходимых сетевой и вычислительной инфраструктур для хранения, извлечения и передачи информации.

В настоящее время доставка рекламного контента до конечного пользователя является сложным многоэтапным процессом, 
определяющим какой конкретный экземпляр рекламы будет показан пользователю в зависимости от его местоположения,
используемых им устройства и программного обеспечения для выхода во Всемирную сеть, истории предыдущих
посещений и прочей известной о пользователе информации.

Алгоритмическая реклама (англ. Programmatic Advertising) является наиболее современным способом реализации 
рекламного содержимого. Принцип алгоритмической рекламы сводится к гранулярности и автоматизации. Гранулярность
обеспечивает быструю обработку доставки рекламного содержимого в реальном времени, позволяет оценивать и ранжировать
стоимость показа и создает условия для оптимизации бюджетной эффективность всех сторон, участвующих в процессе 
реализации рекламного содержимого. Автоматизация открывает двери для детализированной работы с каждым событием доставки,
позволяя персонализировать и актуализировать рекламное содержимое для конечного пользователя.

Прогнозирование взаимодействий пользователей с рекламным содержимым интернет страниц является необходимым этапом
в оценке прибыльности процесса реализации рекламного содержимого. На основе предположения о 
пользовательском поведении, параметры кампаний могут быть исправлены для достижения большей
финансовой эффективности.

В данной работе описан алгоритм решения задачи прогнозирования взаимодействия пользователей с рекламным 
содержимым интернет-страниц. Проведена оценка качества полученного решения в сравнении с существующим методом
наивного сезонного прогнозирования (Na\"{i}ve seasonal forecasting). 

Разработанный алгоритм представляет собой комбинацию методов машинного обучения, 
статистики и вычислительной математики, реализованную средствами программирования, библиотек распределенных
вычислений и современных средств обработки и анализа данных. В качестве основного метода прогнозирования 
используется известный метод наивного сезонного прогнозирования с автоматическим определением периода 
сезонности (Automation Seasonality Detection) на основе анализа спектральной плотности (Power Spectral Density).
Результаты такого прогноза взвешиваются на основе одномерного прогноза набора метрик путем калибровки
(Post--stratification). Оценка полученных результатов производится на различном наборе исходных данных (Back-testing).

Целью работы является разработка алгоритма прогнозирования взаимодействий пользователей с рекламным содержимым
интернет-страниц и оценка качества работы разработанного решения.

Задачи исследования:
\begin{enumerate}
    \item анализ процесса взаимодействия пользователей с рекламными элементами интернет-страниц;
    \item математическая постановка задачи прогнозирования;
    \item разработка алгоритма прогнозирования взаимодействий пользователей;
    \item программная реализация разработанного алгоритма;
    \item оценка качества работы разработанного алгоритма в сравнении с существующим 
    методом наивного сезонного прогнозирования.
\end{enumerate}