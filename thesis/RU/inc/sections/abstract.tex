\structuresection{Реферат}
Пояснительная записка 62 с., 24 рисунка, 2 таблицы, 38 источников, 1 приложение.

Графическая документация: 20 слайдов презентации.

АЛГОРИТМИЧЕСКАЯ РЕКЛАМА, ПРОГНОЗИРОВАНИЕ ВРЕМЕННЫХ РЯДОВ, МЕТОД КАЛИБРОВКИ ВЫБОРКИ, СИСТЕМА ОБРАБОТКИ
БОЛЬШИХ ДАННЫХ

Целью работы являляется построение алгоритма прогнозирования пользовательских взаимодействий
с рекламными элементами интернет-страниц и разработка программной имплементации данного
алгоритма.

В ходе выполенения работы был разработан и реализован алгоритм прогнозирования взаимодействий
пользователей с рекламными элементами интернет страниц. Описаны и математически формализованы
метрики, описывающие пользовательские взаимодействия с рекламными элементами интернет-страниц.

Для имплементации решения использованы современные библиотеки обработки больших данных 
и машинного обучения.

Полученная модель может быть использована для прогнозирования пользовательских взаимодействий
любой рекламной системы, при сведении входных данных к необходимой форме.

