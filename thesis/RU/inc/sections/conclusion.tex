\structuresection{Заключение}

В соответствии с полученным заданием был произведен обзор и анализ предметной области
алгоритмической рекламы, рассмотрены метрики оценки пользовательского поведения и определена
их математическая форма записи, произведен обзор существующих методов решения данной проблемы и
методов, используемых при построении алгоритма.

Перед построением алгоритма, была сформулирована задача, построена модульная математическая модель, 
целью которой является построение прогнозов взаимодействий пользователей, для имплементации 
данной модели были проанализированы и выбраны необходимые средства разработки и библиотеки.

Разработанный метод протестирован методами обратного тестирования на основе реальных данных.
Для оценки качества работы алгоритма было произведено сравнение с существующим методом наивного
сезонного прогнозирования.

Результаты работы данного алгоритма показывают, что разработанный программный комплекс решает
поставленную задачу точнее, чем существующий метод.

Реализация алгоритма произведена при помощи современных библиотек с открытым
исходным кодом, что делает разработанное решение легко поддерживаемым и обслуживаемым 
средством решения поставленной задачи.

Результаты работы будут опубликованы в сборнике конференции ПИТ 2019 (на рассмотрении) в
рамках статьи <<Алгоритм прогнозирования взаимодействия пользователей с рекламными элементами
интернет-страниц>>.