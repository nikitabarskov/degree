\makeatletter
\pgfdeclareshape{datastore}{
  \inheritsavedanchors[from=rectangle]
  \inheritanchorborder[from=rectangle]
  \inheritanchor[from=rectangle]{center}
  \inheritanchor[from=rectangle]{base}
  \inheritanchor[from=rectangle]{north}
  \inheritanchor[from=rectangle]{north east}
  \inheritanchor[from=rectangle]{east}
  \inheritanchor[from=rectangle]{south east}
  \inheritanchor[from=rectangle]{south}
  \inheritanchor[from=rectangle]{south west}
  \inheritanchor[from=rectangle]{west}
  \inheritanchor[from=rectangle]{north west}
  \backgroundpath{
    \southwest \pgf@xa=\pgf@x \pgf@ya=\pgf@y
    \northeast \pgf@xb=\pgf@x \pgf@yb=\pgf@y
    \pgfpathmoveto{\pgfpoint{\pgf@xa}{\pgf@ya}}
    \pgfpathlineto{\pgfpoint{\pgf@xb}{\pgf@ya}}
    \pgfpathmoveto{\pgfpoint{\pgf@xa}{\pgf@yb}}
    \pgfpathlineto{\pgfpoint{\pgf@xb}{\pgf@yb}}
 }
}
\makeatother


\tikzstyle{source} = [
    draw,
    thick,
    rounded corners,
    fill=yellow!20,
    minimum width=4cm, 
    minimum height=1cm,
    text centered, 
    text width=3.75cm]
\tikzstyle{process} = [
    draw,
    thick,
    circle,
    fill=blue!20,
    text centered, 
    text width=2.5cm]
\tikzstyle{datastore} = [
    draw,
    very thick,
    shape=datastore,
    minimum width=4cm, 
    minimum height=1cm,
    text centered, 
    text width=3.75cm]

\begin{figure}[h!]
    \centering
    \begin{tikzpicture}[font=\small]
        \node (source_data) [source] {Исходный протокол работы рекламной системы};
        \node (hadoop_storage) [datastore, right of=source_data, xshift=4cm] {Хранилище исходных данных};
        \draw [arrow] (source_data) -- (hadoop_storage);
        \node (converting) [process, right of=hadoop_storage, xshift=4cm] {Конвертация данных в формат для обработки};
        \draw [arrow] (hadoop_storage) -- (converting);
%         \node (extract) [entity, right of=source_data, xshift=5cm] {Приведение данных к виду, пригодному для обработки};
%         \node (export_report) [entity, right of=extract, xshift=5cm] {Приведение данных к виду, пригодному для обработки};
        % \node (user_2) [entity, below of=user_1, yshift=-0.5cm] {Посетитель 2};
        % \node (user_3) [entity, below of=user_2, yshift=-0.5cm] {Посетитель 3};
%         % \node (page) [entity, right of=user_2, xshift=2.5cm] {Интернет страница с рекламным элементом};
%         % \draw [arrow] (user_1) -| node[anchor=south, font=\small, text width=3cm] {Уведомление о показе рекламного места} (page);
%         % \draw [arrow] (user_2) -- (page);
%         % \draw [arrow] (user_3) -| (page);

%         % \node (ssp) [entity, right of=page, xshift=2.5cm] {SSP};
            
%         % \draw [arrow] (page) -- (ssp);
            
%         % \node (dsp_2) [entity, right of=ssp, xshift=2.5cm] {DSP 2};
%         % \node (dsp_1) [entity, above of=dsp_2, yshift=0.5cm] {DSP 1};
%         % \node (dsp_3) [entity, below of=dsp_2, yshift=-0.5cm] {DSP 3};

%         % \draw [arrow] (ssp) |- node[anchor=south, font=\small, text width=3cm] {Предоставление рекламного места для продажи} (dsp_1);
%         % \draw [arrow] (ssp) -- (dsp_2);
%         % \draw [arrow] (ssp) |- (dsp_3);

%         % \node (advertiser_1) [entity, right of=dsp_1, xshift=2.5cm, yshift=1cm] {Рекламодатель 1};
%         % \node (advertiser_2) [entity, right of=dsp_2, xshift=2.5cm] {Рекламодатель 2};
%         % \node (advertiser_3) [entity, right of=dsp_3, xshift=2.5cm, yshift=-1cm] {Рекламодатель 3};

%         % \draw [arrow] (dsp_1) |- node[anchor=south, font=\small, text width=3cm] {Запрос рекламного содержимого} (advertiser_1);
%         % \draw [arrow] (dsp_2) -- (advertiser_2);
%         % \draw [arrow] (dsp_3) |- (advertiser_3);
    \end{tikzpicture}
    \caption{Конвейер потока данных алгоритма прогнозирования}\label{img:pragrammatic-advertising}
\end{figure}
