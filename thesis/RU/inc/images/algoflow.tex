\tikzstyle{startstop} = [
    rounded rectangle,
    minimum width=4cm,
    minimum height=1cm,
    text centered,
    text width=3.75cm,
    draw=black, 
    fill=red!20]

\tikzstyle{io} = [
    trapezium,
    trapezium left angle=70, 
    trapezium right angle=110, 
    minimum width=4cm, 
    minimum height=1cm, 
    text centered, 
    text width=3.75cm,
    draw=black, 
    fill=blue!20]

\tikzstyle{process} = [
    rectangle, 
    minimum width=4cm,
    minimum height=1cm, 
    text centered,
    text width=3.75cm,
    draw=black, 
    fill=yellow!20]

\medskip
\begin{figure}[h!]
    \centering
    \begin{tikzpicture}[font=\small]
    
        \node [startstop] (start) {Инициализация};
        \node [io, below of=start, yshift=-1cm] (logs) {Исходный протокол работы};
        \node [process, below of=logs, yshift=-1cm] (sample) {Построение вероятностной выборки};
        \node [process, below of=sample, yshift=-1cm] (reports) {Построение статистики метрик};
        \node [process, below of=reports, yshift=-1cm,] (naive) {Построение прогноза протокола работы};
        \node [process, below of=naive, yshift=-1cm] (forecast) {Построение прогноза метрик};
        \node [process, below of=forecast, yshift=-1cm] (calib) {Калибровочный процесс};
        \node [io, below of=calib, yshift=-1cm] (results) {Прогноз пользовательских взаимодействий};
        \node [startstop, below of=results, yshift=-1cm] (end) {Конец};

        \draw [arrow] (start) -- (logs);
        \draw [arrow] (logs) -- (sample);
        \draw [arrow] (sample) -- (reports);
        \draw [arrow] (reports) -- (naive);
        \draw [arrow] (naive) -- (forecast);
        \draw [arrow] (forecast) -- (calib);
        \draw [arrow] (calib) -- (results);
        \draw [arrow] (results) -- (end);
    \end{tikzpicture}
    \caption{Блок-схема разрабатываемого алгоритма прогнозирования}\label{img:algo-flowchart}
\end{figure}